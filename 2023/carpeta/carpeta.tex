\documentclass[a4paper,12pt]{article}
\usepackage[utf8]{inputenc}
\usepackage[T1]{fontenc}
\usepackage[spanish]{babel}
\usepackage{parskip}
\usepackage{float}
\usepackage{enumerate}
\usepackage[cm]{fullpage}
\usepackage{cite}
\usepackage{multirow}
%%%%
% transparent
\usepackage{graphicx}
\usepackage{transparent}
% end transparent
%%%%

\usepackage{colortbl} % para que funcione color en las tablas
\usepackage{tikz}     % para que funcione color en las tablas
\usepackage{array}    % para que tabular funcione centrado
\newcolumntype{C}[1]{>{\centering\arraybackslash}m{#1}}
\newcolumntype{R}[1]{>{\raggedleft\arraybackslash}m{#1}}
\usepackage[left=1.5cm,right=1.5cm, top=1cm, bottom=1cm]{geometry}

\usepackage{subcaption}

%\input{../../aux/tex/encabezado.tex}


\newcommand{\C}[0]{\cellcolor{blue!30}} % definicion del color para tabla

%opening
\title{Cronograma}
\author{\vspace{-0.8cm}\\ Bayes Plurinacional}
\date{}


\usepackage{ragged2e} %\justifying

\newcommand\Wider[2][3em]{%
\makebox[\linewidth][c]{%
  \begin{minipage}{\dimexpr\textwidth+#1\relax}
  \raggedright#2
  \end{minipage}%
  }%
}
\usepackage[absolute,overlay]{textpos} %no funciona
\setlength{\TPHorizModule}{1mm} %128mm  mitad: 64
\setlength{\TPVertModule}{1mm}	%96mm  mitad 48


\begin{document}
\pagenumbering{gobble}


\phantom{.}
\vspace{-1.3cm}

\scalebox{3.5}{\huge \textbf{Bayes}}

\scalebox{3.5}{\huge \textbf{Plurinacional}}

\centering


%
%     \vspace{1.5cm}
%
%     \vfill
%
%     \begin{figure}[h]
%     \hspace{2.5cm}\begin{subfigure}[b]{0.35\textwidth}
%     \centering
%     \includegraphics[width=\linewidth]{../../aux/logos/NODOSdE}
%     \end{subfigure}
%     \hspace{2cm}
%     \begin{subfigure}[b]{0.24\textwidth}
%     \centering
%     \includegraphics[width=\linewidth]{../../aux/logos/secretariaSE.png}
%     \end{subfigure}
%     \end{figure}
%
%     \vspace{1cm}
%
%     \includegraphics[width=0.8\linewidth]{../../aux/logos/beca.png}



%
% \includegraphics[width=0.6\textwidth]{../../aux/logos/BayesPlurinacional.png}
%


\centering
\begin{textblock}{180}(30,107) \centering
\LARGE  \textcolor{black!95}{\rotatebox[origin=tr]{0}{\scalebox{30}{\scalebox{1}[-1]{$p$}}}}
\end{textblock}
\begin{textblock}{180}(30,80) \centering
\LARGE \textcolor{black!95}{\scalebox{30}{$p$}}
\end{textblock}
\begin{textblock}{210}(-70,109) \centering
\LARGE \textcolor{black!95}{\scalebox{14.5}{$($}}
\end{textblock}


\begin{textblock}{190}(10,260) \centering
\begin{figure}[H]
    \centering
    \hspace{0.5cm}\begin{subfigure}[c]{0.2\textwidth}
    \centering
    \includegraphics[width=\linewidth]{../../aux/logos/NODOSdE}
    \end{subfigure}
    \hspace{0.1cm}
    \begin{subfigure}[c]{0.17\textwidth}
    \centering
    \includegraphics[width=\linewidth]{../../aux/logos/secretariaSE.png}
    \end{subfigure}
    \hspace{0.1cm}
    \begin{subfigure}[c]{0.24\textwidth}
    \centering
    \includegraphics[width=\linewidth]{../../aux/logos/sadosky.jpg}
    \end{subfigure}
    \hspace{0.1cm}
    \begin{subfigure}[c]{0.22\textwidth}
    \centering \phantom{.} \vspace{0.6cm}

    \includegraphics[width=\linewidth]{../../aux/logos/LMB.png}
    \end{subfigure}
\end{figure}
\end{textblock}

\newpage






\centering

\scalebox{2.5}{\Large Cronograma}


\begin{textblock}{200}(5,45) \centering


\scalebox{1.4}{\large Viernes 4 de Agosto}

\fontfamily{ptm}\selectfont

\large
\begin{tabular}{ccc}
\hline
\rowcolor{gray!20} \textbf{Horario} &  \hspace{2.8cm} \textbf{Actividad} \hspace{2.8cm}  &  \phantom{.}\hspace{2.9cm} \textbf{Lugar} \hspace{2.9cm} \\
\hline \hline
 8:20 - 9:30 & Acreditación y desayuno & De Entrada a Sala Capacitaciones \\
\hline
\rowcolor{gray!10}
9:30 - 10:10 & Bienvenida y ``Fuente de ideas'' & Sala Capacitaciones a Salas Grupos \\
\hline
10:15 - 11:00 & Taller 1. Intro y evaluación de modelo & Sala Capacitaciones \\
\hline
\rowcolor{gray!10}
11:05 - 12:25 & Charla Bloque 1 (ver detalle) & Sala Capacitaciones \\
\hline
\hline
 12:30 - 13:10 & Almuerzo & Sala Emprendedores \\
 13:15 - 13:45 & Ideas ``Principios, medios y fines'' & Verde Entrada \\
\hline
\hline
\rowcolor{gray!10}
13:50 - 14:40 & Taller 2. Inferencia causal. & Sala Capacitaciones \\
\hline
14:45 - 15:05 & Posters & Pasillo \\
\hline
\rowcolor{gray!10}
15:10 - 16:30 & Charla Bloque 2 (ver detalle) & Sala Capacitaciones \\
\hline

\end{tabular}

\end{textblock}



\begin{textblock}{200}(5,135) \centering


\scalebox{1.4}{\large Sábado 5 de Agosto}

\fontfamily{ptm}\selectfont

\large
\begin{tabular}{ccc}
\hline
\rowcolor{gray!20} \textbf{Horario} &  \hspace{2.8cm} \textbf{Actividad} \hspace{2.8cm}  &  \phantom{.}\hspace{2.9cm} \textbf{Lugar} \hspace{2.9cm} \\
\hline \hline
9:30 - 10:20 & Taller 3. Información y dato. & Sala Capacitaciones \\
\hline
\rowcolor{gray!10}
10:25 - 10:45 & Ideas ``Objetivos, medios y estrategias'' & Sala Capacitaciones \\
\hline
10:50 - 12:10 & Charla Bloque 3 (ver detalle) & Sala Capacitaciones \\
\hline
\hline
\rowcolor{gray!10}
 12:20 - 13:00 & Almuerzo & Sala Emprendedores \\
\rowcolor{gray!10}
 13:05 - 13:35 & Ideas ``Comunidad y colaboración'' & Verde Entrada \\
\hline
\hline
13:40 - 14:30 & Taller 4. Series de tiempo. & Sala Capacitaciones \\
\hline
\rowcolor{gray!10}
14:35 - 14:55 & Posters & Pasillo \\
\hline
15:00 - 16:20 & Charla Bloque 4 (ver detalle) & Sala Capacitaciones \\
\hline
\rowcolor{gray!10}
16:25 - 17:15 & Taller 5. Toma de decisiones y hackatón & Sala Capacitaciones \\
\hline
17:20 - 18:00 & Conclusiones ``Acciones a futuro'' & Sala Capacitaciones \\
\hline
\end{tabular}

\end{textblock}



\begin{textblock}{200}(5,235) \centering


\scalebox{1.4}{\large Domingo 6 de Agosto}

\fontfamily{ptm}\selectfont

\large
\begin{tabular}{ccc}
\hline
\rowcolor{gray!20} \textbf{Horario} &  \hspace{2.8cm} \textbf{Actividad} \hspace{2.8cm}  &  \phantom{.}\hspace{2.9cm} \textbf{Lugar} \hspace{2.9cm} \\
\hline \hline
& & \\
12:30 -  & Almuerzo y música en vivo & El patio del indio Froilán \\
  &  &  \\
\hline
\end{tabular}

\end{textblock}


\newpage

\scalebox{1.7}{\Large Charlas}

\justify \vspace{0.6cm}

\Wider[-2cm]{
\justify
\scalebox{1.2}{\Large Bloque 1}

\vspace{0.2cm}

\hspace{0.6cm} 1.  \textbf{Gabriel Alzamendi}: Métodos basados en inferencia Bayesiana para la evaluación de los mecanísmos vocales de la fonación humana.

\hspace{0.6cm} 2. \textbf{Nicolás Alejandro Comay}: Abordaje Bayesiano a la toma de decisiones perceptuales y su confianza 	asociada.

\hspace{0.6cm} 3. \textbf{Javier Arellana}: Bayesian retrieval scheme and its application to Cassini Mission SAR observations in an area of Titan, Saturn's largest moon.


\vspace{0.6cm}

\scalebox{1.2}{\Large Bloque 2}

\vspace{0.2cm}



\hspace{0.6cm} 1. \textbf{Yasmín Elena Navarrete Díaz}: Sistemas frágiles: una formulación bayesiana de la teoría cuántica.


\hspace{0.6cm} 2. \textbf{Diego Javier Ramón Sevilla}: Regresión bayesiana para el análisis de series temporales con apilamiento en Astronomía de rayos X.

\hspace{0.6cm} 3. \textbf{Andrea Paula Goijman}: Modelos de ocupación multiespecie con enfoque bayesiano: una herramienta valiosa para estudios de biodiversidad en ecología

\justify \vspace{0.6cm}

\scalebox{1.2}{\Large Bloque 3}

\vspace{0.2cm}


\hspace{0.6cm} 1. \textbf{Hossein Dinani}: Bayesian estimation for quantum sensing.


\hspace{0.6cm} 2. \textbf{Albert Ortiz}: Aplicación de métodos bayesianos en ingeniería estructural.

\hspace{0.6cm} 3. \textbf{Gonzalo Ríos}: Bayesian Machine Learning Applied to Marketing.


\justify \vspace{0.6cm}

\scalebox{1.2}{\Large Bloque 4}

\vspace{0.2cm}

\hspace{0.6cm} 1. \textbf{Sergio Michael Davis Irarrázabal}: Probabilidad bayesiana a partir de la estimación plausible de cantidades.

\hspace{0.6cm} 2. \textbf{Tomás Olego}: Métodos robustos Bayesianos Generalizados para los modelos de regresión logística.

\hspace{0.6cm} 3. \textbf{Luciano Moffatt}. Modelado Bayesiano  Computacional de Sistemas Biológicos.
}

\vspace{1cm}

\centering
\scalebox{1.7}{\Large Taller}

\texttt{https://github.com/BayesPlurinacional/tallerBP-2023}

\vspace{0.4cm}

\justify

 \small
\hspace{0.6cm} 1. \textbf{Intro y evaluación de modelo}: Acuerdos intersubjetivos bajo incertidumbre. Especificación gráfica y evaluación de modelos causales. El sobreajuste y el balance natural de las reglas de la probabilidad.

\hspace{0.6cm} 2. \textbf{Inferencia causal}: Inferencia asociacional, intervencional, contrafactual. Identificación de modelo causal. Flujos de inferencia en modelos causales. Conclusiones causales a partir de datos observables.

\hspace{0.6cm} 3. \textbf{Información y datos}: El problema de la comunicación con la realidad. La estructura invariante del dato empírico. Tasa de información. Construcción de sistemas de información (o datos).

\hspace{0.6cm} 4. \textbf{Series de tiempo}: Redes bayesianas de historia completa. Algoritmo de inferencia por pasaje de mensajes. Loopy belief propagation. Inferencia causal en series temporales.

\hspace{0.6cm} 5. \textbf{Toma de decisiones}: La función de costo epistémico-evolutiva. Apuestas óptimas en el tiempo. Ventaja a favor de la diversificación, cooperación, especialización y heterogeneidad. Hackatón.

\newpage

% \scalebox{2.5}{\Large Taller}
%
% \begin{textblock}{220}(-5,10) \centering
% \includegraphics[width=\linewidth]{img/curso-1.png}
% \end{textblock}
%
%
% \begin{textblock}{210}(0,130) \centering
% \includegraphics[width=\linewidth]{img/curso-2.png}
% \end{textblock}
%
%
% \newpage



\phantom{.}
\begin{textblock}{60}(5,3) \centering
\transparent{0.1}{\includegraphics[width=\linewidth]{../../aux/logos/BP.png}}
\end{textblock}
\begin{textblock}{40}(165,270) \centering
\transparent{0.1}{\includegraphics[width=\linewidth]{../../aux/logos/NODOSdE.png}}
\end{textblock}




\newpage


\phantom{.}
\begin{textblock}{60}(5,3) \centering
\transparent{0.1}{\includegraphics[width=\linewidth]{../../aux/logos/BP.png}}
\end{textblock}
\begin{textblock}{35}(170,263) \centering
\transparent{0.1}{\includegraphics[width=\linewidth]{../../aux/logos/secretariaSE.png}}
\end{textblock}

\newpage

\phantom{.}
\begin{textblock}{60}(5,3) \centering
\transparent{0.1}{\includegraphics[width=\linewidth]{../../aux/logos/BP.png}}
\end{textblock}
\begin{textblock}{45}(165,275) \centering
\transparent{0.1}{\includegraphics[width=\linewidth]{../../aux/logos/sadosky.jpg}}
\end{textblock}

\newpage


\phantom{.}
\begin{textblock}{60}(5,3) \centering
\transparent{0.1}{\includegraphics[width=\linewidth]{../../aux/logos/BP.png}}
\end{textblock}
\begin{textblock}{40}(165,266) \centering
\transparent{0.2}{\includegraphics[width=\linewidth]{../../aux/logos/LMB.png}}
\end{textblock}




\end{document}
