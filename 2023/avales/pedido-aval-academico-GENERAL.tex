\documentclass[a4paper,11pt]{article}
\usepackage{paracol}
\usepackage[spanish]{babel}
\usepackage[utf8x]{inputenc}
\usepackage{a4wide}
\usepackage{url}
\usepackage{todonotes}
\usepackage{graphicx}
\usepackage{enumerate}
\usepackage{anysize}
% Controla los m\'argenes {izquierda}{derecha}{arriba}{abajo}. 
\marginsize{2cm}{2cm}{2cm}{2cm}


\usepackage{lipsum}
\usepackage{wallpaper}
\usepackage{fancyhdr}

\pagestyle{fancy}
\fancyhf{}
\renewcommand\headrule{}
\renewcommand{\footrulewidth}{0.4pt}

\fancyhead[RE,LO]{\includegraphics[width=2cm]{../../aux/static/BayesPlurinacional.png} \\
\Large \textbf{Congreso\\ Bayesiano \\Plurinacional}}
% \fancyfoot[CE,CO]{
% % \hline%
% % \vspace{0.1cm}
% \footnotesize Comunidad Bayesiana Plurinacional - Argentina \\
% }
% \fancyfoot[LE,RO]{\vspace{0.2cm}  \thepage}

\setlength\headheight{3.5cm}

\begin{document}
\begin{flushright}
Argentina, 1 de septiembre de 2022
\end{flushright} 

\vspace{0.1cm}
\noindent
Facultad de Ciencias Exactas y Naturales,\\
Universidad de Buenos Aires.\\[-0.1cm]

\hfill \textbf{Ref:} Solicitud de aval institucional al Congreso Bayesiano Plurinacional \\

\vspace{0.3cm} \noindent Estimados miembros de la comisión evaluadora:\\

\indent Un grupo de investigadores de todo el país, reunidos en torno a la aplicación del enfoque bayesiano de la teoría probabilidad, nos dirigimos a ustedes para solicitarles un aval institucional a la organización del Primer Congreso Bayesiano Plurinacional, a realizarse los días 4 y 5 de Agosto del año 2023 en el nodo tecnológico de La Banda, Santiago del Estero, Argentina. \\

\indent Si bien la aplicación estricta de la teoría de la probabilidad (aka ``enfoque bayesiano'') ha mostrado ser la lógica ideal en contextos de incertidumbre, su adopción se vio históricamente limitada debido al alto costo computacional asociado (evaluación de todo el espacio de hipótesis). Aunque en las últimas décadas estas limitaciones han sido superadas en gran medida gracias al cómputo de alto rendimiento y el desarrollo de métodos eficientes de aproximación, la inercia histórica es ahora su limitación principal. \\

\indent  El Congreso tiene por objetivo hacer crecer las comunidades Bayesianas en Argentina y la Región, reuniendo a estudiantes, docentes, investigadores, empresarios, judiciales, practicantes y expertos que utilizan, desarrollan o implementan métodos Bayesianos. Entre los tópicos a tratar se incluye:, modelado probabilístico, inferencia causal, toma de decisiones, programación probabilística, teoría de la información, inteligencia artificial, ciencia de datos, epistemología y metodología, estadística aplicada, pedagogía de la probabilidad. \\

\indent Sin otro particular, la comisión organizadora del CBP les saluda a ustedes muy atentamente.\\


 \vspace{1.1cm}


 \begin{paracol}{3}

  \scriptsize

%\includegraphics[width=4cm]{firma.png}\\[-0.4cm]
\noindent\line(1,0){120}\\
Dr. Osvaldo Martín \\
Investigador Conicet \\
Universidad de San Luis


 \switchcolumn

%\includegraphics[width=4cm]{firma.png}\\[-0.4cm]
\noindent\line(1,0){120}\\
Dr. Rodrigo Díaz \\
Investigador Conicet \\
Universidad Nacional de San Martín\\

 \switchcolumn

%\includegraphics[width=4cm]{firma.png}\\[-0.4cm]
\noindent\line(1,0){120}\\
Lic. María Gimena Serrano \\
Secretaría de CyT, Sgo. del Estero \\
Universidad de Tucuman

 \end{paracol}

 \vspace{1.1cm}


\begin{paracol}{3}

  \scriptsize

%\includegraphics[width=4cm]{firma.png}\\[-0.4cm]
\noindent\line(1,0){120}\\
Lic. Gustavo Andrés Landfried \\
Doctorando en Cs de la Computación \\
Universidad de Buenos Aires


 \switchcolumn

%\includegraphics[width=4cm]{firma.png}\\[-0.4cm]
\noindent\line(1,0){120}\\
Dr. ¿Ariel Hernán Curiale? \\
Investigador Conicet \\
Centro Atómico Bariloche - Balseiro

 \switchcolumn

%\includegraphics[width=4cm]{firma.png}\\[-0.4cm]
\noindent\line(1,0){120}\\
Lic. Ariel Silvio Norberto Ramos\\
Profesión \\
Universidad de Salta

 \end{paracol}


 \vspace{1.1cm}


\begin{paracol}{3}

  \scriptsize

%\includegraphics[width=4cm]{firma.png}\\[-0.4cm]
\noindent\line(1,0){120}\\
Lic. ¿Federico Carrone?\\
Empresa LambdaClass \\
Universidad de Buenos Aires

 \switchcolumn

%\includegraphics[width=4cm]{firma.png}\\[-0.4cm]
\noindent\line(1,0){120}\\
Dr. --- \\
 --- \\
---

 \switchcolumn

%\includegraphics[width=4cm]{firma.png}\\[-0.4cm]
\noindent\line(1,0){120}\\
Lic. --- \\
--- \\
---

 \end{paracol}


\end{document}
