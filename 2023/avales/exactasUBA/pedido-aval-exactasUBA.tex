\documentclass[a4paper,11pt]{article}
\usepackage{paracol}
\usepackage[spanish]{babel}
\usepackage[utf8x]{inputenc}
\usepackage{a4wide}
\usepackage{url}
\usepackage{hyperref}
\usepackage{todonotes}
\usepackage{graphicx}
\usepackage{enumerate}
\usepackage{anysize}
% Controla los m\'argenes {izquierda}{derecha}{arriba}{abajo}.
\marginsize{2cm}{2cm}{1cm}{2cm}
\usepackage{textpos}
\setlength{\TPHorizModule}{1mm}
\setlength{\TPVertModule}{1mm}


\hypersetup{
    colorlinks=true,
    linkcolor={red!50!black},
    citecolor={blue!35!black},
    urlcolor={blue!35!black}
}

\usepackage{lipsum}
\usepackage{wallpaper}
\usepackage{fancyhdr}

\pagestyle{fancy}
\fancyhf{}
\renewcommand\headrule{}
\renewcommand{\footrulewidth}{0.4pt}

\fancyhead[RE,LO]{
%\includegraphics[width=2cm]{../../../aux/static/cbp.png}
\begin{textblock}{160}(24,1.45)
\noindent \LARGE\textbf{Congreso\\ Bayesiano \\Plurinacional \\ }
\end{textblock}
% \begin{textblock}{160}(24,1.43)
% \noindent \LARGE\textbf{Plurinational \\ Bayesian \\ Congress\\ }
% \end{textblock}
\begin{textblock}{80}(6.5,3.4)
\LARGE  {\rotatebox[origin=tr]{0}{\scalebox{5}{\scalebox{1}[-1]{$p$}}}}
\end{textblock}
\begin{textblock}{80}(6.5,0)
\LARGE {\scalebox{5}{$p$}}
\end{textblock}
\begin{textblock}{80}(0,3.6)
\LARGE {\scalebox{2.505}{$($}}
\end{textblock}
\phantom{\includegraphics[width=2cm]{../../../aux/static/cbp.png}}
}
\setlength\headheight{3.5cm}

\begin{document}
\begin{flushright}
Argentina, 2 de noviembre del 2022
\end{flushright} 

\vspace{0.1cm}
\noindent
Universidad de Buenos Aires \\
Facultad de Ciencias Exactas y Naturales \\
Decano Dr. Guillermo Durán\\[-0.1cm]

\hfill \textbf{Ref:} Solicitud de aval institucional al Congreso Bayesiano Plurinacional \\

\vspace{0.3cm} \noindent Consejo Directivo:\\

\indent Un grupo de investigadores de todo el país, reunidos en torno a la aplicación del enfoque bayesiano de la probabilidad, nos dirigimos a ustedes para solicitarles un aval institucional a la organización del primer Congreso Bayesiano Plurinacional, \url{bayesdelsur.com.ar}, a realizarse los días 4 y 5 de agosto del año 2023 en el nodo tecnológico de La Banda, Santiago del Estero, Argentina. \\

\indent La aplicación estricta de las reglas de la probabilidad ha mostrado ser la lógica ideal para todas las ciencias empíricas.
Sin embargo su adopción se vio históricamente limitada debido al alto costo computacional que requiere la evaluación completa del espacio de hipótesis.
Aunque en las últimas décadas estas limitaciones han sido superadas en gran medida gracias al desarrollo de métodos eficientes de aproximación, la inercia histórica es ahora su obstáculo principal. \\

\indent  El Congreso tiene por objetivo hacer crecer las comunidades bayesianas en Argentina y la región, reuniendo a estudiantes, docentes, investigadores, emprendedores y practicantes que trabajan con el enfoque bayesiano de la probabilidad. Entre los tópicos a tratar se incluye modelado probabilístico, inferencia causal, toma de decisiones, programación probabilística, teoría de la información, inteligencia artificial, ciencia de datos, epistemología y metodología, entre otros. \\

\indent En caso de aceptación incluiremos el logo de la institución en la página web del Congreso. Sin otro particular, la comisión organizadora les saluda a ustedes muy atentamente.\\

 \vspace{0.3cm}

 \begin{paracol}{3}

  \scriptsize

%\includegraphics[width=4cm]{firma.png}\\[-0.4cm]
%\noindent\line(1,0){120}\\
\noindent Dr. Osvaldo Martín \\
Investigador Conicet \\
IMASL-CONICET (San Luis)


 \switchcolumn

   %\includegraphics[width=4cm]{firma.png}\\[-0.4cm]
%\noindent\line(1,0){120}\\
\noindent Dr. Rodrigo Díaz \\
Investigador Conicet \\
Universidad Nacional de San Martín\\

 \switchcolumn

%\includegraphics[width=4cm]{firma.png}\\[-0.4cm]
%\noindent\line(1,0){120}\\
\noindent Lic. María Gimena Serrano \\
Secretaría de Ciencia y Técnica \\
Gobierno de Santiago del Estero \\

 \end{paracol}


\begin{paracol}{3}

  \scriptsize


%\includegraphics[width=4cm]{firma.png}\\[-0.4cm]
%\noindent\line(1,0){120}\\
\noindent  Lic. Gustavo Landfried \\
Doctorando en Cs de la Computación \\
Universidad de Buenos Aires

 \switchcolumn

%\includegraphics[width=4cm]{firma.png}\\[-0.4cm]
%\noindent\line(1,0){120}\\
\noindent  Lic. Martina Cántaro \\
Project Manager \\
LambdaClass

 \switchcolumn

%\includegraphics[width=4cm]{firma.png}\\[-0.4cm]
%\noindent\line(1,0){120}\\
\noindent Tec. Ariel Silvio Norberto Ramos\\
Developer independiente \\
Universidad Nacional de Salta

 \end{paracol}

 \vspace{1cm}

Investigadores de CONICET que apoyan la organización del Congreso (lista no exhaustiva).

\vspace{0.3cm}


  \scriptsize
 \noindent  Rodolfo Guillermo Pregliasco (Instituto Balseiro, Bariloche), Jorge A. Vila (Instituo de Matemática Aplicada de San Luis), Silvia Schiaffino (Instituto Superior de Ingeniería del Software, Tandíl), Inés Samengo (Centro Atómico Bariloche e Instituo Balseiro), Mariano De Paula (Centro de Investigaciones en Física e Ingeniería, Tandíl), Gabriel Alejandro Alzamendi (Instituto de Investigaciones y Desarrollo en Bioingeniería y Bioinformatica, Entre Ríos), Cristina Noemi Gardenal (Instituto de Diversidad y Ecología Animal, Córdoba), Alejandro Olivieri (Instituo de Quimica Rosario, Universidad Nacional de Rosario), Viviana Andrea Confalonieri (Instituo de Ecología, Genética y Evolución de Buenos Aires), Gabriel Montes Rojas (Instituto Interdisciplinario de Economía Política, Buenos Aires), Inés Caridi (Instituto de Cálculo, Buenos Aires), Leandro Luna (Instituo Multidisciplinario de Historia y Ciencias Humanas, Buenos Aires), Gastón Bujía (Instituto de Cálculo, Buenos Aires), Reinaldo Andrés	Moralejo (Facultad de Ciencias Naturales y Museo, La Plata), Adela Tisnés (Instituto de Geografía, Historia y Ciencias Sociales, Tandíl), Marcelo Gabriel Armentano (Instituto Superior de Ingeniería del Software, Tandil), Luciano Moffat (Instituo de Quimica, Física de los Materiales, Medioambiente y Energía, Buenos Aires), Hernán Curiale (Comisión Nacional de Energía Atómica, Bariloche), Veronica Gil-Costa (Universidad Nacional de San Luis), Germán Ezequiel Lescano (Universidad Nacional de Santiago del Estero).%Ferando Buezas (Instituto de Física del Sur, Bahía Blanca),

\end{document}
