\documentclass[a4paper,11pt]{article}
\usepackage{paracol}
\usepackage[spanish]{babel}
\usepackage[utf8x]{inputenc}
\usepackage{a4wide}
\usepackage{url}
\usepackage{hyperref}
\usepackage{todonotes}
\usepackage{graphicx}
\usepackage{enumerate}
\usepackage{anysize}
% Controla los m\'argenes {izquierda}{derecha}{arriba}{abajo}. 
\marginsize{2cm}{2cm}{2cm}{2cm}
\usepackage{textpos}
\setlength{\TPHorizModule}{1mm}
\setlength{\TPVertModule}{1mm}


\hypersetup{
    colorlinks=true,
    linkcolor={red!50!black},
    citecolor={blue!35!black},
    urlcolor={blue!35!black}
}

\usepackage{lipsum}
\usepackage{wallpaper}
\usepackage{fancyhdr}

\pagestyle{fancy}
\fancyhf{}
\renewcommand\headrule{}
\renewcommand{\footrulewidth}{0.4pt}

\fancyhead[RE,LO]{
\begin{textblock}{160}(24,5)
\noindent \LARGE\textbf{Plurinational \\ Bayesian \\ Congress \\ }
\end{textblock}
\includegraphics[width=2cm]{../../../aux/static/BayesPlurinacional2.png}
}
% \fancyfoot[CE,CO]{
% % \hline%
% % \vspace{0.1cm}
% \footnotesize Comunidad Bayesiana Plurinacional - Argentina \\
% }
% \fancyfoot[LE,RO]{\vspace{0.2cm}  \thepage}

\setlength\headheight{3.5cm}

\begin{document}
\begin{flushright}
Comisión organizadora del CBP \\
Argentina, 1 de septiembre de 2022
\end{flushright} 

\vspace{0.1cm}
\noindent
APELLIDO-NOMBRE\\
CARGO \\
Miembro del Consejo Nacional de Investigaciones Científicas y Técnicas.\\

\hfill \textbf{Ref:} Invitación de apoyo al Congreso Bayesiano Plurinacional \\

\noindent APELLIDO-NOMBRE:\\

\indent Un grupo de investigadores de todo el país, reunidos en torno a la aplicación del enfoque bayesiano de la teoría probabilidad, nos dirigimos a usted para invitarle personalmente a participar, activa o pasivamente, de la organización del primer Congreso Bayesiano Plurinacional, a realizarse los días 4 y 5 de Agosto del año 2023 en las instalaciones del NODO tecnológico de Santiago del Estero, Argentina. \\

\indent Si bien la aplicación estricta de la teoría de la probabilidad (enfoque bayesiano) ha mostrado ser la lógica ideal en contextos de incertidumbre, su adopción se vio históricamente limitada debido al alto costo computacional asociado (evaluación de todo el espacio de hipótesis). Aunque en las últimas décadas estas limitaciones han sido superadas en gran medida gracias al cómputo de alto rendimiento y el desarrollo de métodos eficientes de aproximación, la inercia histórica es ahora su limitación principal. \\

\indent El Congreso tiene por objetivo hacer crecer las comunidades Bayesianas en Argentina y la Región, reuniendo a estudiantes, docentes, investigadores, emprendedores, practicantes que utilizan, desarrollan o implementan métodos Bayesianos en sus respectivos trabajos. Entre los tópicos a tratar se incluye: modelado probabilístico, razonamiento causal, toma de decisiones, programación probabilística, teoría de la información, inteligencia artificial, ciencia de datos, epistemología y metodología, estadística aplicada, pedagogía de la probabilidad. \\

\indent Los tipos de apoyo a la organización del Congreso pueden ser: simbólica (aportando el nombre como aval); activa a través del contacto directo con la organización (enviando un correo a ). Se adjunta la primera circular informativa. \\

\indent Sin otro particular, le saluda a usted muy atentamente la

\textbf{comisión organizadora del primer Congreso Bayesiano Plurinacional}\\


 \vspace{0.6cm}

 \begin{paracol}{3}

  \scriptsize

%\includegraphics[width=4cm]{firma.png}\\[-0.4cm]
%\noindent\line(1,0){120}\\
\noindent Dr. Osvaldo Martín \\
Investigador Conicet \\
IMASL-CONICET (San Luis)


 \switchcolumn

   %\includegraphics[width=4cm]{firma.png}\\[-0.4cm]
%\noindent\line(1,0){120}\\
\noindent Dr. Rodrigo Díaz \\
Investigador Conicet \\
Universidad Nacional de San Martín\\

 \switchcolumn

%\includegraphics[width=4cm]{firma.png}\\[-0.4cm]
%\noindent\line(1,0){120}\\
\noindent Lic. María Gimena Serrano \\
Secretaría de Ciencia y Técnica \\
Gobierno de Santiago del Estero \\

 \end{paracol}

 \vspace{0.6cm}


\begin{paracol}{3}

  \scriptsize


%\includegraphics[width=4cm]{firma.png}\\[-0.4cm]
%\noindent\line(1,0){120}\\
\noindent  Lic. Gustavo Landfried \\
Doctorando en Cs de la Computación \\
Universidad de Buenos Aires



 \switchcolumn

%\includegraphics[width=4cm]{firma.png}\\[-0.4cm]
%\noindent\line(1,0){120}\\
\noindent  Dr. Ariel Hernán Curiale \\
Investigador Conicet \\
Centro Atómico Bariloche - Balseiro

 \switchcolumn

%\includegraphics[width=4cm]{firma.png}\\[-0.4cm]
%\noindent\line(1,0){120}\\
\noindent Tec. Ariel Silvio Norberto Ramos\\
Developer independiente \\
Universidad de Salta

 \end{paracol}

 \vspace{0.8cm}
\small
 Contacto: \texttt{bayesdelsur@gmail.com}

 Sitio web: \texttt{bayesdelsur.com.ar}


\end{document}
