\documentclass{article}
\usepackage[english]{babel}
\usepackage{tikz}
\usepackage{caption}
\usepackage{float} % para que los gr\'aficos se queden en su lugar con [H]
\usepackage{subcaption}

\usepackage[utf8]{inputenc}
\usepackage{url}
\usepackage{graphicx}
\usepackage{color}
\usepackage{amsmath} %para escribir funci\'on partida , matrices
\usepackage{amsthm} %para numerar definciones y teoremas
\usepackage[hidelinks]{hyperref} % para inlcuir links dentro del texto
%\usepackage{tabu} 
\usepackage{comment}
\usepackage{amsfonts} % \mathbb{N} -> conjunto de los n\'umeros naturales  
\usepackage{enumerate}
\usepackage{listings}
\usepackage[colorinlistoftodos, textsize=small]{todonotes} % Para poner notas en el medio del texto!! No olvidar hacer. 
\usepackage{framed} % Para encuadrar texto. \begin{framed}
\usepackage{csquotes} % Para citar texto \begin{displayquote}
\usepackage{varioref}
\usepackage{bm} % \bm{\alpha} bold greek symbol
\usepackage{pdfpages} % \includepdf
\usepackage[makeroom]{cancel} % \cancel{} \bcancel{} etc
\usepackage{wrapfig} % \begin{wrapfigure} Pone figura al lado del texto
\usepackage{mdframed}
\usepackage{algorithm}
\usepackage{quoting}
\usepackage{mathtools}	
\usepackage{paracol}

\newcommand{\vm}[1]{\mathbf{#1}}
\newcommand{\N}{\mathcal{N}}
\newcommand{\citel}[1]{\cite{#1}\label{#1}}
\newcommand\hfrac[2]{\genfrac{}{}{0pt}{}{#1}{#2}} %\frac{}{} sin la linea del medio

\newtheorem{midef}{Definition}
\newtheorem{miteo}{Theorem}
\newtheorem{mipropo}{Proposition}

\theoremstyle{definition}
\newtheorem{definition}{Definition}[section]
\newtheorem{theorem}{Theorem}[section]
\newtheorem{proposition}{Proposition}[section]


%http://latexcolor.com/
\definecolor{azul}{rgb}{0.36, 0.54, 0.66}
\definecolor{rojo}{rgb}{0.7, 0.2, 0.116}
\definecolor{rojopiso}{rgb}{0.8, 0.25, 0.17}
\definecolor{verdeingles}{rgb}{0.12, 0.5, 0.17}
\definecolor{ubuntu}{rgb}{0.44, 0.16, 0.39}
\definecolor{debian}{rgb}{0.84, 0.04, 0.33}
\definecolor{dkgreen}{rgb}{0,0.6,0}
\definecolor{gray}{rgb}{0.5,0.5,0.5}
\definecolor{mauve}{rgb}{0.58,0,0.82}

\lstset{
  language=Python,
  aboveskip=3mm,
  belowskip=3mm,
  showstringspaces=true,
  columns=flexible,
  basicstyle={\small\ttfamily},
  numbers=none,
  numberstyle=\tiny\color{gray},
  keywordstyle=\color{blue},
  commentstyle=\color{dkgreen},
  stringstyle=\color{mauve},
  breaklines=true,
  breakatwhitespace=true,
  tabsize=4
}

% tikzlibrary.code.tex
%
% Copyright 2010-2011 by Laura Dietz
% Copyright 2012 by Jaakko Luttinen
%
% This file may be distributed and/or modified
%
% 1. under the LaTeX Project Public License and/or
% 2. under the GNU General Public License.
%
% See the files LICENSE_LPPL and LICENSE_GPL for more details.

% Load other libraries

%\newcommand{\vast}{\bBigg@{2.5}}
% newcommand{\Vast}{\bBigg@{14.5}}
% \usepackage{helvet}
% \renewcommand{\familydefault}{\sfdefault}

\usetikzlibrary{shapes}
\usetikzlibrary{fit}
\usetikzlibrary{chains}
\usetikzlibrary{arrows}

% Latent node
\tikzstyle{latent} = [circle,fill=white,draw=black,inner sep=1pt,
minimum size=20pt, font=\fontsize{10}{10}\selectfont, node distance=1]
% Observed node
\tikzstyle{obs} = [latent,fill=gray!25]
% Invisible node
\tikzstyle{invisible} = [latent,minimum size=0pt,color=white, opacity=0, node distance=0]
% Constant node
\tikzstyle{const} = [rectangle, inner sep=0pt, node distance=0.1]
%state
\tikzstyle{estado} = [latent,minimum size=8pt,node distance=0.4]
%action
\tikzstyle{accion} =[latent,circle,minimum size=5pt,fill=black,node distance=0.4]
\tikzstyle{fijo} =[latent,circle,minimum size=5pt,fill=black]


% Factor node
\tikzstyle{factor} = [rectangle, fill=black,minimum size=10pt, draw=black, inner
sep=0pt, node distance=1]
% Deterministic node
\tikzstyle{det} = [latent, rectangle]

% Plate node
\tikzstyle{plate} = [draw, rectangle, rounded corners, fit=#1]
% Invisible wrapper node
\tikzstyle{wrap} = [inner sep=0pt, fit=#1]
% Gate
\tikzstyle{gate} = [draw, rectangle, dashed, fit=#1]

% Caption node
\tikzstyle{caption} = [font=\footnotesize, node distance=0] %
\tikzstyle{plate caption} = [caption, node distance=0, inner sep=0pt,
below left=5pt and 0pt of #1.south east] %
\tikzstyle{factor caption} = [caption] %
\tikzstyle{every label} += [caption] %

\tikzset{>={triangle 45}}

%\pgfdeclarelayer{b}
%\pgfdeclarelayer{f}
%\pgfsetlayers{b,main,f}

% \factoredge [options] {inputs} {factors} {outputs}
\newcommand{\factoredge}[4][]{ %
  % Connect all nodes #2 to all nodes #4 via all factors #3.
  \foreach \f in {#3} { %
    \foreach \x in {#2} { %
      \path (\x) edge[-,#1] (\f) ; %
      %\draw[-,#1] (\x) edge[-] (\f) ; %
    } ;
    \foreach \y in {#4} { %
      \path (\f) edge[->,#1] (\y) ; %
      %\draw[->,#1] (\f) -- (\y) ; %
    } ;
  } ;
}

% \edge [options] {inputs} {outputs}
\newcommand{\edge}[3][]{ %
  % Connect all nodes #2 to all nodes #3.
  \foreach \x in {#2} { %
    \foreach \y in {#3} { %
      \path (\x) edge [->,#1] (\y) ;%
      %\draw[->,#1] (\x) -- (\y) ;%
    } ;
  } ;
}

% \factor [options] {name} {caption} {inputs} {outputs}
\newcommand{\factor}[5][]{ %
  % Draw the factor node. Use alias to allow empty names.
  \node[factor, label={[name=#2-caption]#3}, name=#2, #1,
  alias=#2-alias] {} ; %
  % Connect all inputs to outputs via this factor
  \factoredge {#4} {#2-alias} {#5} ; %
}

% \plate [options] {name} {fitlist} {caption}
\newcommand{\plate}[4][]{ %
  \node[wrap=#3] (#2-wrap) {}; %
  \node[plate caption=#2-wrap] (#2-caption) {#4}; %
  \node[plate=(#2-wrap)(#2-caption), #1] (#2) {}; %
}

% \gate [options] {name} {fitlist} {inputs}
\newcommand{\gate}[4][]{ %
  \node[gate=#3, name=#2, #1, alias=#2-alias] {}; %
  \foreach \x in {#4} { %
    \draw [-*,thick] (\x) -- (#2-alias); %
  } ;%
}

% \vgate {name} {fitlist-left} {caption-left} {fitlist-right}
% {caption-right} {inputs}
\newcommand{\vgate}[6]{ %
  % Wrap the left and right parts
  \node[wrap=#2] (#1-left) {}; %
  \node[wrap=#4] (#1-right) {}; %
  % Draw the gate
  \node[gate=(#1-left)(#1-right)] (#1) {}; %
  % Add captions
  \node[caption, below left=of #1.north ] (#1-left-caption)
  {#3}; %
  \node[caption, below right=of #1.north ] (#1-right-caption)
  {#5}; %
  % Draw middle separation
  \draw [-, dashed] (#1.north) -- (#1.south); %
  % Draw inputs
  \foreach \x in {#6} { %
    \draw [-*,thick] (\x) -- (#1); %
  } ;%
}

% \hgate {name} {fitlist-top} {caption-top} {fitlist-bottom}
% {caption-bottom} {inputs}
\newcommand{\hgate}[6]{ %
  % Wrap the left and right parts
  \node[wrap=#2] (#1-top) {}; %
  \node[wrap=#4] (#1-bottom) {}; %
  % Draw the gate
  \node[gate=(#1-top)(#1-bottom)] (#1) {}; %
  % Add captions
  \node[caption, above right=of #1.west ] (#1-top-caption)
  {#3}; %
  \node[caption, below right=of #1.west ] (#1-bottom-caption)
  {#5}; %
  % Draw middle separation
  \draw [-, dashed] (#1.west) -- (#1.east); %
  % Draw inputs
  \foreach \x in {#6} { %
    \draw [-*,thick] (\x) -- (#1); %
  } ;%
}



\newcommand{\congresoNombre}{Congreso Sudamericano de Análisis Bayesiano}
\newcommand{\congresoSigla}{CSAB}

\title{\congresoNombre \, (\congresoSigla) \\ y Taller Argentino de Computación Científica (TACC)}

\author{Bayes de la Provincias Unidas del Sur}
\affil[]{\small Correspondencia: \texttt{bayesdelsur, aloctavodia, gustavolandfried @ gmail.com}}


\begin{document}
  
\maketitle

\begin{abstract}
Las ciencias empíricas, a diferencia de las ciencias formales, deben validar sus proposiciones dentro de sistemas abiertos que contienen incertidumbre.
%
Si bien la teoría de la probabilidad (Bayesiana) preserva los acuerdos intersubjetivos en contextos de incertidumbre, fundamento de las verdades empíricas, su adopción se vio históricamente limitada a pesar de ser conceptualmente sencilla.
%
La organización del primer \congresoNombre \,  (\congresoSigla) junto al Taller Argentino de Computación Científica (TACC) tiene por objetivo reunir a personas que transitan diferentes etapas de formación, aplicación, desarrollo, e innovación, de modo de producir un círculo virtuoso entre ellas.
%
El calendario de actividades se propone producir la interacción entre las partes a través de la implementación de: cursos, hackatón, experiencias y reflexión.
\end{abstract}

\begin{figure}[ht!]
 \centering
\tikz{
    \node[accion] (n0) {} ;
    \node[const, left=of n0] (nn0) {$n_0$} ;
    \node[accion, yshift=-0.33cm, xshift=0.33cm] (n1) {} ;
    \node[const, below=of n1] (nn1) {$n_1$} ;
    
    \node[accion, yshift=-0.33cm, xshift=3.33cm] (n2) {} ;
    \node[accion, xshift=3.66cm] (n3) {} ;
    \node[const, below=of n2] (nn2) {$n_2$} ;
    \node[const, right=of n3] (nn3) {$n_3$} ;
    
    \node[accion, yshift=2.49444cm , xshift=2.1cm] (n4) {} ;
    \node[accion, yshift=2.49444cm , xshift=1.6333cm] (n5) {} ;
    \node[const, right=of n4] (nn4) {$n_4$} ;
    \node[const, left=of n5] (nn5) {$n_5$} ;
    
    
    %\edge {n1} {n2}
    \path[draw, ->, fill=black!50,sloped] (n1) edge[draw=black!50] node[midway,below,color=black!75] {\scriptsize  Hakatón} (n2);
    
    \path[draw, ->, fill=black!50,sloped] (n5) edge[draw=black!50] node[midway,above,color=black!75] {\scriptsize  Curso} (n0);
    
    \path[draw, ->, fill=black!50,sloped] (n3) edge[draw=black!50] node[midway,above,color=black!75] {\scriptsize  Experiencias} (n4);
    
    \path[draw, ->, fill=black!50,sloped] (n4) edge[bend right, in=-90, out=-90, looseness=3, draw=black!50] node[midway,above,color=black!75] {\scriptsize  Reflexión} (n5);
    
    %\node[const, yshift=-0.51cm, xshift=-0.33cm] (na) {\rotatebox[origin=tr]{-45}{}} ;

    }
    \caption{}
\end{figure}

El TACC está compuesto por Cursos y Hackatón.

El \congresoSigla \ está compuesto por Expriencias y Reflexiones

\vspace{0.3cm} %%%%%%%%%%%%%%%%%%%%%%%%%%%%%%%%%%%%%%%%%%%%%%

Las personas que participan:
%
\begin{itemize}\setlength\itemsep{-0.1cm}
\item[$n_0:$] No conocen la metodología bayesiana (estudiantes, investigadores de otras áreas) 
\item[$n_1:$] Quieren usar la metodología bayesiana (estudiantes, programadores, doctorandos) 
\item[$n_2:$] Tienen para resolver problemas concretos (investigadores, empresas) 
\item[$n_3:$] Resolvieron problemas usando PPLs (investigadores, empresas, doctorandos) 
\item[$n_4:$] Desarrollan los PPLs (developers, investigadores) 
\item[$n_5:$] Implementan o investigan algoritmos en detalle (developers, investigadores) 
\end{itemize}
%
Las personas pueden identificarse en más de una de las categorías mencionadas.
Las actividades simultaneas que realiza el \congresoSigla \ y el TACC son:

\begin{itemize}\setlength\itemsep{-0.1cm}
\item \textbf{Curso - Experiencia}. 
En estos momentos solo se ven obligados a dividirse los $n_4$ de los $n_5$.
Los $n_1$ están libres para participar del curso.
Y los $n_2$ están libres para participar de las experiencias.
O para descansar.

\item \textbf{Hackaton - Reflexión}
En estos momentos no hay divisiones por grupos.
Los $n_0$ y $n_3$ están libres para participar del hakatón, o para descansar.

\item \textbf{Plenarios (o fiestas)}
Participan todas.

\end{itemize}




\end{document}

% A diferencia de las ciencias formales, las ciencias empíricas deben validar sus proposiciones dentro de sistemas abiertos, lo que impone siempre un grado de incertidumbre asociada.
% %
% La teoría de la probabilidad (Bayesiana) permite validar verdades empíricas gracias a que garantiza la preservación de los acuerdos intersubjetivos dada la evidencia empírica y formal (datos y modelos causales).
% %
% Aunque sea conceptualmente sencilla, su adopción en las ciencias empíricas se vio históricamente limitada debido a que su aplicación suele presentar desafíos matemáticos y computacionales importantes.
% %
% Con la masificación del poder de cómputo el enfoque Bayesiano comenzó a aplicarse para resolver cada vez más problemas hasta ser considerado, en la era de la inteligencia artificial, el marco teórico de referencia.
% %
% Si bien todavía existen desafíos no resueltos, los métodos existentes en la actualidad permiten implementar 
% 
