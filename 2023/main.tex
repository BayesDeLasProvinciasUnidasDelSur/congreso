\documentclass{article}
\input{../aux/tex/encabezado.tex}
\input{../aux/tex/tikzlibrarybayesnet.code.tex}

\newcommand{\congresoNombre}{Congreso Sudamericano de Análisis Bayesiano}
\newcommand{\congresoSigla}{CSAB}

\title{\congresoNombre \, (\congresoSigla) \\ y Taller Argentino de Computación Científica (TACC)}

\author{Bayes de la Provincias Unidas del Sur}
\affil[]{\small Correspondencia: \texttt{bayesdelsur, aloctavodia, gustavolandfried @ gmail.com}}


\begin{document}
  
\maketitle

\begin{abstract}
Las ciencias empíricas, a diferencia de las ciencias formales, deben validar sus proposiciones dentro de sistemas abiertos que contienen incertidumbre.
%
Si bien la teoría de la probabilidad (Bayesiana) preserva los acuerdos intersubjetivos en contextos de incertidumbre, fundamento de las verdades empíricas, su adopción se vio históricamente limitada a pesar de ser conceptualmente sencilla.
%
La organización del primer \congresoNombre \,  (\congresoSigla) junto al Taller Argentino de Computación Científica (TACC) tiene por objetivo reunir a personas que transitan diferentes etapas de formación, aplicación, desarrollo, e innovación, de modo de producir un círculo virtuoso entre ellas.
%
El calendario de actividades se propone producir la interacción entre las partes a través de la implementación de: cursos, hackatón, experiencias y reflexión.
\end{abstract}

\begin{figure}[ht!]
 \centering
\tikz{
    \node[accion] (n0) {} ;
    \node[const, left=of n0] (nn0) {$n_0$} ;
    \node[accion, yshift=-0.33cm, xshift=0.33cm] (n1) {} ;
    \node[const, below=of n1] (nn1) {$n_1$} ;
    
    \node[accion, yshift=-0.33cm, xshift=3.33cm] (n2) {} ;
    \node[accion, xshift=3.66cm] (n3) {} ;
    \node[const, below=of n2] (nn2) {$n_2$} ;
    \node[const, right=of n3] (nn3) {$n_3$} ;
    
    \node[accion, yshift=2.49444cm , xshift=2.1cm] (n4) {} ;
    \node[accion, yshift=2.49444cm , xshift=1.6333cm] (n5) {} ;
    \node[const, right=of n4] (nn4) {$n_4$} ;
    \node[const, left=of n5] (nn5) {$n_5$} ;
    
    
    %\edge {n1} {n2}
    \path[draw, ->, fill=black!50,sloped] (n1) edge[draw=black!50] node[midway,below,color=black!75] {\scriptsize  Hakatón} (n2);
    
    \path[draw, ->, fill=black!50,sloped] (n5) edge[draw=black!50] node[midway,above,color=black!75] {\scriptsize  Curso} (n0);
    
    \path[draw, ->, fill=black!50,sloped] (n3) edge[draw=black!50] node[midway,above,color=black!75] {\scriptsize  Experiencias} (n4);
    
    \path[draw, ->, fill=black!50,sloped] (n4) edge[bend right, in=-90, out=-90, looseness=3, draw=black!50] node[midway,above,color=black!75] {\scriptsize  Reflexión} (n5);
    
    %\node[const, yshift=-0.51cm, xshift=-0.33cm] (na) {\rotatebox[origin=tr]{-45}{}} ;

    }
    \caption{}
\end{figure}

El TACC está compuesto por Cursos y Hackatón.

El \congresoSigla \ está compuesto por Expriencias y Reflexiones

\vspace{0.3cm} %%%%%%%%%%%%%%%%%%%%%%%%%%%%%%%%%%%%%%%%%%%%%%

Las personas que participan:
%
\begin{itemize}\setlength\itemsep{-0.1cm}
\item[$n_0:$] No conocen la metodología bayesiana (estudiantes, investigadores de otras áreas) 
\item[$n_1:$] Quieren usar la metodología bayesiana (estudiantes, programadores, doctorandos) 
\item[$n_2:$] Tienen para resolver problemas concretos (investigadores, empresas) 
\item[$n_3:$] Resolvieron problemas usando PPLs (investigadores, empresas, doctorandos) 
\item[$n_4:$] Desarrollan los PPLs (developers, investigadores) 
\item[$n_5:$] Implementan o investigan algoritmos en detalle (developers, investigadores) 
\end{itemize}
%
Las personas pueden identificarse en más de una de las categorías mencionadas.
Las actividades simultaneas que realiza el \congresoSigla \ y el TACC son:

\begin{itemize}\setlength\itemsep{-0.1cm}
\item \textbf{Curso - Experiencia}. 
En estos momentos solo se ven obligados a dividirse los $n_4$ de los $n_5$.
Los $n_1$ están libres para participar del curso.
Y los $n_2$ están libres para participar de las experiencias.
O para descansar.

\item \textbf{Hackaton - Reflexión}
En estos momentos no hay divisiones por grupos.
Los $n_0$ y $n_3$ están libres para participar del hakatón, o para descansar.

\item \textbf{Plenarios (o fiestas)}
Participan todas.

\end{itemize}




\end{document}

% A diferencia de las ciencias formales, las ciencias empíricas deben validar sus proposiciones dentro de sistemas abiertos, lo que impone siempre un grado de incertidumbre asociada.
% %
% La teoría de la probabilidad (Bayesiana) permite validar verdades empíricas gracias a que garantiza la preservación de los acuerdos intersubjetivos dada la evidencia empírica y formal (datos y modelos causales).
% %
% Aunque sea conceptualmente sencilla, su adopción en las ciencias empíricas se vio históricamente limitada debido a que su aplicación suele presentar desafíos matemáticos y computacionales importantes.
% %
% Con la masificación del poder de cómputo el enfoque Bayesiano comenzó a aplicarse para resolver cada vez más problemas hasta ser considerado, en la era de la inteligencia artificial, el marco teórico de referencia.
% %
% Si bien todavía existen desafíos no resueltos, los métodos existentes en la actualidad permiten implementar 
% 
