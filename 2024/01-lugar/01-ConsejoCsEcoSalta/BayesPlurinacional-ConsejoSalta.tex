\documentclass[a4paper,10pt]{letter}
\usepackage{paracol}
\usepackage[spanish]{babel}
\usepackage[utf8x]{inputenc}
\usepackage{a4wide}
\usepackage{url}
\usepackage{hyperref}
\usepackage{todonotes}
\usepackage{graphicx}
\usepackage{enumerate}
\usepackage{anysize}
% Controla los m\'argenes {izquierda}{derecha}{arriba}{abajo}.
\marginsize{2cm}{2cm}{1cm}{2cm}
\usepackage{textpos}
\setlength{\TPHorizModule}{1mm}
\setlength{\TPVertModule}{1mm}
 \usepackage{pdflscape}
\usepackage{geometry}
%%%%
% transparent
\usepackage{graphicx}
\usepackage{transparent}
% end transparent
%%%%


\hypersetup{
    colorlinks=true,
    linkcolor={red!50!black},
    citecolor={blue!35!black},
    urlcolor={blue!35!black}
}

\usepackage{lipsum}
\usepackage{wallpaper}
\usepackage{fancyhdr}




\pagenumbering{gobble}
\begin{document}



\begin{letter}

%\includegraphics[width=2cm]{../../../aux/static/cbp.png}
\begin{textblock}{160}(4.5,-30.5)
\noindent \Huge\textbf{Bayes \\ Plurinacional \\}
\end{textblock}
% \begin{textblock}{160}(24,1.43)
% \noindent \LARGE\textbf{Plurinational \\ Bayesian \\ Congress\\ }
% \end{textblock}
\begin{textblock}{80}(-14.5,-30.2)
\LARGE  {\rotatebox[origin=tr]{0}{\scalebox{5}{\scalebox{1}[-1]{$p$}}}}
\end{textblock}
\begin{textblock}{80}(-14.5,-33.6)
\LARGE {\scalebox{5}{$p$}}
\end{textblock}
\begin{textblock}{80}(-22,-30)
\LARGE {\scalebox{2.505}{$($}}
\end{textblock}
\phantom{\includegraphics[width=2cm]{../../../aux/static/cbp.png}}




\vspace{-4.2cm}

\begin{flushright}
Argentina, 12 de diciembre del 2023  \\
\end{flushright}

\noindent

\hfill \textbf{Ref:} Propuesta de realización de la Escuela Bayesiana Plurinacional en Salta \\


\noindent Presidente Cr. Juan Pablo López López \\
\noindent Consejo Profesional de Ciencias Económicas de Salta\\ [0cm]

\hspace{1cm}
La comunidad Bayes Plurinacional ha culminado con éxito la organización del Congreso Bayesiano Plurinacional 2023 los días 4, 5 y 6 de agosto en las instalaciones de la Secretaría de Ciencia y Tecnología de Santiago del Estero.
%
El Gobernador de esa provincia, Gerardo Zamora, nos visitó durante la primera jornada de este evento de relevancia continental.
%
La aplicación estricta de las reglas de la probabilidad (enfoque bayesiano) no sólo es el corazón de la inteligencia artificial, es también el razonamiento común a todas las ciencias empíricas (o ciencias con datos) que deben validar sus proposiciones en contextos de incertidumbre.
%
Esto explica la extraordinaria diversidad disciplinar con la que contamos: ecología, medicina, astronomía, geología, psicología, cuántica, economía, ingeniería, filosofía, biología, estadística, antropología, computación entre otras.
%
Además, el carácter continental del evento se evidencia en la diversidad de regiones: Chile, Paraguay, Uruguay, Colombia, Honduras, Brasil, Argentina (Córdoba, Salta, Mendoza, Entre Rios, Santa Fe, Santiago del Estero, Buenos Aires), del sector privado, público y la academia.

% Parrafo

\hspace{1cm} En todo este tiempo, desde su descubrimiento en el siglo 18, no se ha propuesto un sistema mejor en términos prácticos que la aplicación estricta de las reglas de la probabilidad.
%
El costo computacional asociado a la evaluación de todo el espacio de hipótesis fue un el límite para el enfoque bayesiano hasta la aparición de las computadoras personales en las vísperas del siglo 21.
%
Si bien en la actualidad el crecimiento del enfoque bayesiano es un fenómeno mundial que se está acelerando, su desarrollo sigue siendo incipiente debido a una fuerte inercia histórica.
%
La decisión de realizar la primera Escuela Bayesiana Plurinacional está basada en el hecho de que ni siquiera los principales centros universitarios de nuestro continente garantizan en la actualidad la formación en inferencia bayesiana.
%
En este contexto nos proponemos organizar un encuentro presencial de 3 a 5 días, donde se ofrezcan cursos bayesianos prácticos para los diferentes niveles: iniciales, intermedios y avanzados.

% Parrafo

\hspace{1cm} Impulsar una formación práctica es el principal objetivo de la Escuela Bayesiana Plurinacional.
%
Los métodos bayesianos ofrecen varios beneficios para la economía y son utilizados en diversos contextos para evaluar el conocimiento de forma óptima: incorporan información previa o conocimiento experto en el modelo; expresan y actualizan la incertidumbre maximizando el uso de la información; y permite crear y evaluar correctamente los modelos o argumentos causales alternativos maximizando el poder predictivo.
%
Los métodos bayesianos se utilizan para predecir y analizar patrones temporales en datos económicos, como tasas de interés, tasas de cambio, precios de acciones, etc.
%
Los métodos bayesianos se aplican en la evaluación y gestión de riesgos financieros, como la valoración de opciones, la estimación de VaR (Valor en Riesgo) y la modelización de pérdidas extremas.
%
En finanzas, los métodos bayesianos se utilizan para el análisis de portafolios, permitiendo la incorporación de información previa sobre rendimientos y volatilidades para mejorar las estimaciones y decisiones de asignación de activos.
%
En definitiva, los métodos bayesianos permiten maximizar la tasa de crecimiento de cualquier problema de apuestas en el tiempo.
%
Por esa razón son el sistema de razonamiento para contextos de incertidumbre.

% Parrafo

\hspace{1cm}
Esperamos estrechar nuestras relaciones. Atentamente,

\vspace{.3cm}

\begin{textblock}{160}(0,0)
\phantom{.} \hfill \includegraphics[width=4cm]{../../../aux/img/firma.png}\hspace{2cm}\phantom{.} \\[0cm]
\end{textblock}
\begin{textblock}{160}(0,20)
 \phantom{.} \hfill Gustavo Landfried \hspace{2.5cm}\phantom{.}\\ \small
\phantom{.} \hfill Secrertario General Interino \hspace{2.5cm}\phantom{.}\\
\phantom{.} \hfill Bayes Plurinacional \hspace{2.5cm}\phantom{.}\\
\end{textblock}

%
% \begin{flushright}
% \hfill \includegraphics[width=0.3\linewidth]{firma.png}\hspace{2cm}\phantom{.} \\[0cm]
% \hfill Gustavo Landfried \hspace{2.5cm}\phantom{.}\\ \small
% \hfill Secrertario General Interino \hspace{2.5cm}\phantom{.}\\
% \hfill de Bayes Plurinacional \hspace{2.5cm}\phantom{.}\\
% \end{flushright}
%


 \vspace{0.8cm}
\small
 %Contacto: \texttt{bayesdelsur@gmail.com}

%  Sitio web: \texttt{www.bayesdelsur.com.ar}



 \end{letter}


\end{document}
