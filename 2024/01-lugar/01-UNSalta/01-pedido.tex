\documentclass[a4paper,10pt]{letter}
\usepackage{paracol}
\usepackage[spanish]{babel}
\usepackage[utf8x]{inputenc}
\usepackage{a4wide}
\usepackage{url}
\usepackage{hyperref}
\usepackage{todonotes}
\usepackage{graphicx}
\usepackage{enumerate}
\usepackage{anysize}
% Controla los m\'argenes {izquierda}{derecha}{arriba}{abajo}.
\marginsize{2cm}{2cm}{1cm}{2cm}
\usepackage{textpos}
\setlength{\TPHorizModule}{1mm}
\setlength{\TPVertModule}{1mm}
 \usepackage{pdflscape}
\usepackage{geometry}
%%%%
% transparent
\usepackage{graphicx}
\usepackage{transparent}
% end transparent
%%%%


\hypersetup{
    colorlinks=true,
    linkcolor={red!50!black},
    citecolor={blue!35!black},
    urlcolor={blue!35!black}
}

\usepackage{lipsum}
\usepackage{wallpaper}
\usepackage{fancyhdr}




\pagenumbering{gobble}
\begin{document}



\begin{letter}

%\includegraphics[width=2cm]{../../../aux/static/cbp.png}
\begin{textblock}{160}(4.5,-30.5)
\noindent \Huge\textbf{Bayes \\ Plurinacional \\}
\end{textblock}
% \begin{textblock}{160}(24,1.43)
% \noindent \LARGE\textbf{Plurinational \\ Bayesian \\ Congress\\ }
% \end{textblock}
\begin{textblock}{80}(-14.5,-30.2)
\LARGE  {\rotatebox[origin=tr]{0}{\scalebox{5}{\scalebox{1}[-1]{$p$}}}}
\end{textblock}
\begin{textblock}{80}(-14.5,-33.6)
\LARGE {\scalebox{5}{$p$}}
\end{textblock}
\begin{textblock}{80}(-22,-30)
\LARGE {\scalebox{2.505}{$($}}
\end{textblock}
\phantom{\includegraphics[width=2cm]{../../../aux/static/cbp.png}}




\vspace{-4.2cm}

\begin{flushright}
Argentina, 4 de septiembre del 2023  \\
\end{flushright}

\noindent

\hfill \textbf{Ref:} Propuesta de realización de la Escuela Bayesiana Plurinacional en la UNSa \\


\noindent Ingeniero Daniel Hoyos \\
\noindent Rector de la Universidad Nacional de Salta\\ [0cm]

\hspace{1cm}
La comunidad Bayes Plurinacional ha culminado con éxito la organización del Congreso Bayesiano Plurinacional 2023 los días 4, 5 y 6 de agosto en las instalaciones de la Secretaría de Ciencia y Tecnología de Santiago del Estero.
%
El Gobernador de esa provincia, Gerardo Zamora, nos visitó durante la primera jornada de este evento de relevancia continental.
%
La aplicación estricta de las reglas de la probabilidad (enfoque bayesiano) no sólo es el corazón de la inteligencia artificial, es también el razonamiento común a todas las ciencias empíricas (o ciencias con datos) que deben validar sus proposiciones en contextos de incertidumbre.
%
Esto explica la extraordinaria diversidad disciplinar con la que contamos: ecología, medicina, astronomía, geología, psicología, cuántica, economía, ingeniería, filosofía, biología, estadística, antropología, computación entre otras.
%
Además, el carácter continental del evento se evidencia en la diversidad de regiones: Chile, Paraguay, Uruguay, Colombia, Honduras, Brasil, Argentina (Córdoba, Salta, Mendoza, Entre Rios, Santa Fe, Santiago del Estero, Buenos Aires), del sector privado, público y la academia.

% Parrafo

\hspace{1cm}
Al finalizar el Congreso la asamblea decidió por unanimidad realizar la primera Escuela Bayesiana Plurinacional entre finales de julio y principios de agosto del 2024.
%
A pesar de que las reglas de la probabilidad se conocen desde finales del siglo 18 y no se ha propuesto desde entonces nada mejor en lo práctico, el costo computacional asociado a la evaluación de todo el espacio de hipótesis ha sido un límite importante para su aplicación hasta las vísperas del siglo 21.
%
Si bien en la actualidad el crecimiento del enfoque bayesiano es un fenómeno mundial que se está acelerando, su desarrollo sigue siendo incipiente debido a una fuerte inercia histórica, especialmente en Latinoamérica.
%
La decisión de realizar la primera Escuela Bayesiana Plurinacional está basada en el hecho de que ni siquiera los principales centros universitarios de nuestro continente garantizan en la actualidad la formación en inferencia bayesiana.
%
En este contexto nos proponemos organizar un encuentro presencial de aproximadamente 5 días, donde se ofrezcan cursos bayesianos prácticos para los diferentes niveles: iniciales, intermedios y avanzados.

% Parrafo

\hspace{1cm}
Durante la asamblea se propusieron varias sedes posibles donde realizar la Escuela.
%
La UNSa fue elegida como primera opción, especialmente por la estrecha relación que la Secretaría de Extensión estableció con la comunidad Bayes Plurinacional a través de Ariel Silvio Norberto Ramos, coordinador del programa de extensión ``Entornos digitales para Educación Permanente y Capacitación Laboral'' (Res. R. Nro $1039$/$22$), y Rubén Emilio Correa, Secretario de SEU-UNSa.
%
Creemos que realizar la primera Escuela Bayesiana Plurinacional en las instalaciones de la UNSa beneficia directamente a toda la comunidad universitaria, en todas sus disciplinas científicas y técnicas.
%
Además, colocaría a la UNSa en una posición de vanguardia continental en una área de relevancia fundamental para todos los campos de la ciencia, poniéndola en contacto directo con las principales figuras de nuestra región.
%
Las ventajas para la comunidad Bayes Plurinacional son: la ubicación en el norte argentino, cercano a varios países limítrofes; las instalaciones de la Universidad que cuenta con salas, campo deportivo, habitaciones para hospedar visitantes entre otros; y el especial interés que ya existe al interior de la Universidad y la provincia de Salta por las actividades de formación en métodos bayesianos gracias al estrecho vínculo que la Secretaría de Extensión ya ha construido con la comunidad Bayes Plurinacional.

% Parrafo

\hspace{1cm}
Esperamos estrechar nuestras relaciones. Atentamente,

\vspace{.3cm}

\begin{textblock}{160}(0,0)
\phantom{.} \hfill \includegraphics[width=4cm]{../../../aux/img/firma.png}\hspace{2cm}\phantom{.} \\[0cm]
\end{textblock}
\begin{textblock}{160}(0,20)
 \phantom{.} \hfill Gustavo Landfried \hspace{2.5cm}\phantom{.}\\ \small
\phantom{.} \hfill Secrertario General Interino \hspace{2.5cm}\phantom{.}\\
\phantom{.} \hfill Bayes Plurinacional \hspace{2.5cm}\phantom{.}\\
\end{textblock}

%
% \begin{flushright}
% \hfill \includegraphics[width=0.3\linewidth]{firma.png}\hspace{2cm}\phantom{.} \\[0cm]
% \hfill Gustavo Landfried \hspace{2.5cm}\phantom{.}\\ \small
% \hfill Secrertario General Interino \hspace{2.5cm}\phantom{.}\\
% \hfill de Bayes Plurinacional \hspace{2.5cm}\phantom{.}\\
% \end{flushright}
%


 \vspace{0.8cm}
\small
 %Contacto: \texttt{bayesdelsur@gmail.com}

%  Sitio web: \texttt{www.bayesdelsur.com.ar}



 \end{letter}


\end{document}
