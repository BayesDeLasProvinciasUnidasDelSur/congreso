\documentclass[shownotes,aspectratio=169]{beamer}

\input{../../aux/tex/diapo_encabezado.tex}
% tikzlibrary.code.tex
%
% Copyright 2010-2011 by Laura Dietz
% Copyright 2012 by Jaakko Luttinen
%
% This file may be distributed and/or modified
%
% 1. under the LaTeX Project Public License and/or
% 2. under the GNU General Public License.
%
% See the files LICENSE_LPPL and LICENSE_GPL for more details.

% Load other libraries

%\newcommand{\vast}{\bBigg@{2.5}}
% newcommand{\Vast}{\bBigg@{14.5}}
% \usepackage{helvet}
% \renewcommand{\familydefault}{\sfdefault}

\usetikzlibrary{shapes}
\usetikzlibrary{fit}
\usetikzlibrary{chains}
\usetikzlibrary{arrows}

% Latent node
\tikzstyle{latent} = [circle,fill=white,draw=black,inner sep=1pt,
minimum size=20pt, font=\fontsize{10}{10}\selectfont, node distance=1]
% Observed node
\tikzstyle{obs} = [latent,fill=gray!25]
% Invisible node
\tikzstyle{invisible} = [latent,minimum size=0pt,color=white, opacity=0, node distance=0]
% Constant node
\tikzstyle{const} = [rectangle, inner sep=0pt, node distance=0.1]
%state
\tikzstyle{estado} = [latent,minimum size=8pt,node distance=0.4]
%action
\tikzstyle{accion} =[latent,circle,minimum size=5pt,fill=black,node distance=0.4]
\tikzstyle{fijo} =[latent,circle,minimum size=5pt,fill=black]


% Factor node
\tikzstyle{factor} = [rectangle, fill=black,minimum size=10pt, draw=black, inner
sep=0pt, node distance=1]
% Deterministic node
\tikzstyle{det} = [latent, rectangle]

% Plate node
\tikzstyle{plate} = [draw, rectangle, rounded corners, fit=#1]
% Invisible wrapper node
\tikzstyle{wrap} = [inner sep=0pt, fit=#1]
% Gate
\tikzstyle{gate} = [draw, rectangle, dashed, fit=#1]

% Caption node
\tikzstyle{caption} = [font=\footnotesize, node distance=0] %
\tikzstyle{plate caption} = [caption, node distance=0, inner sep=0pt,
below left=5pt and 0pt of #1.south east] %
\tikzstyle{factor caption} = [caption] %
\tikzstyle{every label} += [caption] %

\tikzset{>={triangle 45}}

%\pgfdeclarelayer{b}
%\pgfdeclarelayer{f}
%\pgfsetlayers{b,main,f}

% \factoredge [options] {inputs} {factors} {outputs}
\newcommand{\factoredge}[4][]{ %
  % Connect all nodes #2 to all nodes #4 via all factors #3.
  \foreach \f in {#3} { %
    \foreach \x in {#2} { %
      \path (\x) edge[-,#1] (\f) ; %
      %\draw[-,#1] (\x) edge[-] (\f) ; %
    } ;
    \foreach \y in {#4} { %
      \path (\f) edge[->,#1] (\y) ; %
      %\draw[->,#1] (\f) -- (\y) ; %
    } ;
  } ;
}

% \edge [options] {inputs} {outputs}
\newcommand{\edge}[3][]{ %
  % Connect all nodes #2 to all nodes #3.
  \foreach \x in {#2} { %
    \foreach \y in {#3} { %
      \path (\x) edge [->,#1] (\y) ;%
      %\draw[->,#1] (\x) -- (\y) ;%
    } ;
  } ;
}

% \factor [options] {name} {caption} {inputs} {outputs}
\newcommand{\factor}[5][]{ %
  % Draw the factor node. Use alias to allow empty names.
  \node[factor, label={[name=#2-caption]#3}, name=#2, #1,
  alias=#2-alias] {} ; %
  % Connect all inputs to outputs via this factor
  \factoredge {#4} {#2-alias} {#5} ; %
}

% \plate [options] {name} {fitlist} {caption}
\newcommand{\plate}[4][]{ %
  \node[wrap=#3] (#2-wrap) {}; %
  \node[plate caption=#2-wrap] (#2-caption) {#4}; %
  \node[plate=(#2-wrap)(#2-caption), #1] (#2) {}; %
}

% \gate [options] {name} {fitlist} {inputs}
\newcommand{\gate}[4][]{ %
  \node[gate=#3, name=#2, #1, alias=#2-alias] {}; %
  \foreach \x in {#4} { %
    \draw [-*,thick] (\x) -- (#2-alias); %
  } ;%
}

% \vgate {name} {fitlist-left} {caption-left} {fitlist-right}
% {caption-right} {inputs}
\newcommand{\vgate}[6]{ %
  % Wrap the left and right parts
  \node[wrap=#2] (#1-left) {}; %
  \node[wrap=#4] (#1-right) {}; %
  % Draw the gate
  \node[gate=(#1-left)(#1-right)] (#1) {}; %
  % Add captions
  \node[caption, below left=of #1.north ] (#1-left-caption)
  {#3}; %
  \node[caption, below right=of #1.north ] (#1-right-caption)
  {#5}; %
  % Draw middle separation
  \draw [-, dashed] (#1.north) -- (#1.south); %
  % Draw inputs
  \foreach \x in {#6} { %
    \draw [-*,thick] (\x) -- (#1); %
  } ;%
}

% \hgate {name} {fitlist-top} {caption-top} {fitlist-bottom}
% {caption-bottom} {inputs}
\newcommand{\hgate}[6]{ %
  % Wrap the left and right parts
  \node[wrap=#2] (#1-top) {}; %
  \node[wrap=#4] (#1-bottom) {}; %
  % Draw the gate
  \node[gate=(#1-top)(#1-bottom)] (#1) {}; %
  % Add captions
  \node[caption, above right=of #1.west ] (#1-top-caption)
  {#3}; %
  \node[caption, below right=of #1.west ] (#1-bottom-caption)
  {#5}; %
  % Draw middle separation
  \draw [-, dashed] (#1.west) -- (#1.east); %
  % Draw inputs
  \foreach \x in {#6} { %
    \draw [-*,thick] (\x) -- (#1); %
  } ;%
}


 \mode<presentation>
 {
 %   \usetheme{Madrid}      % or try Darmstadt, Madrid, Warsaw, ...
 %   \usecolortheme{default} % or try albatross, beaver, crane, ...
 %   \usefonttheme{serif}  % or try serif, structurebold, ...
  \usetheme{Antibes}
  \setbeamertemplate{navigation symbols}{}
 }
\usetikzlibrary{decorations.text}
\usepackage{rotating}
\usepackage{transparent}

\usepackage{todonotes}
\setbeameroption{show notes}

\newcounter{capitulo}
\setcounter{capitulo}{1}
\newcommand{\unidad}{\thecapitulo \stepcounter{capitulo}}


\estrue

%\title[Bayes del Sur]{}

\begin{document}

\color{black!85}
\large

\begin{frame}[plain,noframenumbering]

\begin{textblock}{160}(0,24) \centering
\LARGE \textcolor{black!85}{\fontsize{22}{0}\selectfont \textbf{Congreso Bayesiano Plurinacional}}
\end{textblock}
\begin{textblock}{160}(0,6) \centering
\LARGE  \textcolor{black!85}{\rotatebox[origin=tr]{0}{\scalebox{4}{\scalebox{1}[-1]{$p$}}}}
\end{textblock}
\begin{textblock}{160}(0,6) \centering
\LARGE \textcolor{black!85}{\scalebox{4}{$p$}}
\end{textblock}
\begin{textblock}{160}(0,6) \centering
\LARGE \textcolor{black!85}{\scalebox{3.7}{C} \hspace{2.4cm} }
\end{textblock}
\begin{textblock}{160}(0,6) \centering
\LARGE \textcolor{black!85}{ \hspace{2.5cm} \scalebox{3.7}{P} }
\end{textblock}

\begin{textblock}{160}(0,40)\centering
\hspace{3.6cm} \LARGE  \textcolor{black!85}{\rotatebox[origin=tr]{-3}{\scalebox{6}{\scalebox{1}[-1]{$p$}}}}
\end{textblock}

\begin{textblock}{160}(0,49) \centering
\LARGE  \textcolor{black!85}{\scalebox{4}{$=$}}
\end{textblock}

\begin{textblock}{160}(0,40)\centering
\hspace{-3.8cm} \LARGE  \textcolor{black!85}{\scalebox{6}{$p$}}
\end{textblock}

\begin{textblock}{160}(0,73) \centering \Large \textcolor{black!75}{\textbf{
Del 4 al 5 de agosto 2023 \\
La Banda, Santiago del Estero, Argentina \\[0.1cm]}}

\normalsize \texttt{bayesdelsur@gmail.com}
\end{textblock}

\end{frame}


\begin{frame}[plain,noframenumbering]

\begin{textblock}{160}(01,03)\centering
\textcolor{black!85}{{\large
Curso virtual\\[-0.1cm] \Large Verdades empíricas \\[-0.1cm] \footnotesize Hacia el Congreso Bayesiano Plurinacional 2023}}
\end{textblock}


\begin{textblock}{150}(10,20)
\normalsize Primera Parte. Fundamentos - 2022. \\
\scriptsize No se requiere ningún tipo de formación previa \\[0.15cm] \footnotesize
\ \ $1$. Principios interculturales de acuerdos intersubjetivos \\
\ \ $2$. La función de costo epistémico-evolutiva\\
\ \ $3$. Sorpresa: el problema de la comunicación con la realidad \\
\ \ $4$. Modelos gráficos y algoritmo suma-producto\\
\ \ $5$. Flujos de inferencia \\
\ \ $6$. Inferencia causal \\[0.4cm]
\normalsize Segunda Parte. Metodologías - 2023.\scriptsize \\ Sin implicar exclusión, se requieren algunos conocimiento mínimos de álgebra, análisis y programación. \\[0.15cm] \footnotesize
\ \ $7$. Distribuciones de creencias \\
\ \ $8$. Evaluación de modelos \\
\ \ $9$. Aproximaciones analíticas \\
\ \ $10$. Series de tiempo\\
\ \ $11$. Aproximaciones por exploración \\
\ \ $12$. Programación probabilística\\
\end{textblock}



\end{frame}


\begin{frame}[plain,noframenumbering]


\begin{textblock}{160}(0,0)
\includegraphics[width=1\textwidth]{../../aux/static/deforestacion}
\end{textblock}

\begin{textblock}{80}(18,9)
\textcolor{black!15}{\fontsize{44}{55}\selectfont Verdades}
\end{textblock}

\begin{textblock}{47}(85,70)
\centering \textcolor{black!15}{{\fontsize{52}{65}\selectfont Empíricas}}
\end{textblock}

\begin{textblock}{80}(100,28)
\LARGE  \textcolor{black!15}{\rotatebox[origin=tr]{-3}{\scalebox{9}{\scalebox{1}[-1]{$p$}}}}
\end{textblock}

\begin{textblock}{80}(66,43)
\LARGE  \textcolor{black!15}{\scalebox{6}{$=$}}
\end{textblock}

\begin{textblock}{80}(36,29)
\LARGE  \textcolor{black!15}{\scalebox{9}{$p$}}
\end{textblock}

\vspace{2cm}
\maketitle



\begin{textblock}{160}(01,81)
\textcolor{black!5}{\textbf{\Large Curso 2022 - 2023 \\
\scriptsize Hacia el Congreso Bayesiano Plurinacional\\}}
\end{textblock}

\end{frame}


\begin{frame}[plain,noframenumbering]
\begin{textblock}{170}(-9,0)
\rotatebox[origin=tr]{90}{\includegraphics[width=0.53\textwidth]{../../aux/static/egipto3.jpeg}}
\end{textblock}

\begin{textblock}{160}(16,9)
\LARGE \textcolor{black!5}{\fontsize{22}{0}\selectfont \textbf{Principios interculturales}}
\end{textblock}
\begin{textblock}{160}(22,18)
\LARGE \textcolor{black!5}{\fontsize{22}{0}\selectfont \textbf{de acuerdos intersubjetivos}}
\end{textblock}


\begin{textblock}{55}(71,38)
\begin{turn}{33}
\parbox{6cm}{
\textcolor{black!5}{\hspace{-0.3cm}Capítulo \unidad} \\
\small\textcolor{black!5}{\hspace{-0.1cm}Principio de razón suficiente, de}\\
\small\textcolor{black!5}{integridad, de indiferencia y de} \\
\small\textcolor{black!5}{\hspace{0.1cm}coherencia. Las reglas de razo-} \\ \small\textcolor{black!5}{\hspace{0.2cm}namiento bajo incertidumbre.} \\
\small\textcolor{black!5}{\hspace{0.36cm}Evaluación de creencias.} \\
}
\end{turn}
\end{textblock}

\end{frame}

\begin{frame}[plain,noframenumbering]

% \begin{textblock}{160}(0,0)
% \includegraphics[width=1.18\textwidth]{../../aux/static/fotosintesis}
% \end{textblock}
\begin{textblock}{160}(0,-15)
\includegraphics[width=1\textwidth]{../../aux/static/tsimane}
\end{textblock}


% VERSION 2
\begin{textblock}{160}(6,36)
\LARGE \rotatebox[origin=tr]{18}{\textcolor{black!95}{\fontsize{22}{0}\selectfont \textbf{La función}}}
\end{textblock}
\begin{textblock}{160}(41,32)
\LARGE \rotatebox[origin=tr]{23}{\textcolor{black!95}{\fontsize{22}{0}\selectfont \textbf{de}}}
\end{textblock}
\begin{textblock}{160}(50.5,23)
\LARGE \rotatebox[origin=tr]{28}{\textcolor{black!95}{\fontsize{22}{0}\selectfont \textbf{costo}}}
\end{textblock}
\begin{textblock}{160}(68,5.3)
\LARGE \rotatebox[origin=tr]{26}{\textcolor{black!95}{\fontsize{22}{0}\selectfont \textbf{epistémico}}}
\end{textblock}
\begin{textblock}{160}(104,5.5)
\LARGE \rotatebox[origin=tr]{8}{\textcolor{black!95}{\fontsize{22}{0}\selectfont \textbf{-}}}
\end{textblock}
\begin{textblock}{160}(110,3)
\LARGE \rotatebox[origin=tr]{-14}{\textcolor{black!95}{\fontsize{22}{0}\selectfont \textbf{evolutiva}}}
\end{textblock}


\begin{textblock}{55}[0,0](120,22)
\begin{turn}{-57}
\parbox{7cm}{\sloppy\setlength\parfillskip{0pt}
\textcolor{black!0}{\ \ \ \ \ Capítulo \unidad} \\
\small\textcolor{black!5}{\hspace{-0.15cm} Ventajas a favor de la:} \\
\small\textcolor{black!5}{\hspace{-1.45cm} Diversificación (propiedad epistémica)}\\
\small\textcolor{black!5}{\hspace{-1.7cm} Cooperación (propiedad evolutiva mayor)}\\
\small\textcolor{black!5}{ \hspace{-1.75cm}Especialización (propiedad meta-epistémica)} \\
\small\textcolor{black!5}{\hspace{-2cm} Coexistencia (propiedad ecológica).\\ }}
\end{turn}
\end{textblock}


\end{frame}

\begin{frame}[plain,noframenumbering]

\begin{textblock}{160}(0,0)
\includegraphics[width=1\textwidth]{../../aux/static/fuego}
\end{textblock}

\begin{textblock}{160}(4,26)
\LARGE \textcolor{black!5}{\fontsize{22}{0}\selectfont \textbf{Sorpresa: el problema}}
\end{textblock}
\begin{textblock}{160}(4,34)
\LARGE \textcolor{black!5}{\fontsize{22}{0}\selectfont \textbf{de la comunicación}}
\end{textblock}
\begin{textblock}{160}(4,42)
\LARGE \textcolor{black!5}{\fontsize{22}{0}\selectfont \textbf{con la realidad}}
\end{textblock}
% \begin{textblock}{160}(3,82)
% \LARGE \textcolor{black!15}{\fontsize{22}{0}\selectfont \textbf{3}}
% \end{textblock}



\begin{textblock}{55}[0,0](88,25)
\begin{turn}{0}
\parbox{7cm}{\sloppy\setlength\parfillskip{0pt}
\textcolor{black!0}{Capítulo \unidad} \\
\small\textcolor{black!5}{\hspace{0.05cm}La estructura invariante del dato empírico:} \\
\small\textcolor{black!5}{\hspace{0.1cm}fuente, realidad causal, señal, canal,} \\ \small\textcolor{black!5}{\hspace{0.05cm}percepción, modelo causal, estimación.} \\
\small\textcolor{black!5}{\hspace{-0.15cm}Base empírica y datos teóricos. Máxima} \\
\small\textcolor{black!5}{\hspace{-0.35cm}incertidumbre y mínima sorpresa. Información.} \\
}
\end{turn}
\end{textblock}


\end{frame}

\begin{frame}[plain,noframenumbering]
\begin{textblock}{160}(0,43)
\includegraphics[width=1\textwidth]{../../aux/static/modelosGraficos}
\end{textblock}


\begin{textblock}{160}(4,4)
\LARGE \textcolor{black!85}{\fontsize{22}{0}\selectfont \textbf{Modelos gráficos y}}
\end{textblock}
\begin{textblock}{160}(4,12)
\LARGE \textcolor{black!85}{\fontsize{22}{0}\selectfont \textbf{algoritmo suma-producto}}
\end{textblock}


\begin{textblock}{55}[0,0](70,27)
\begin{turn}{0}
\parbox{10cm}{\sloppy\setlength\parfillskip{0pt}
\textcolor{black!85}{Capítulo \unidad} \\
\small\textcolor{black!85}{Métodos gráficos de especificación de modelos causales.} \\
\small\textcolor{black!85}{Cómputo descentralizado de la inferencia y la predicción:} \\
\small\textcolor{black!85}{pasaje de mensajes entre los nodos de las redes causales}\\
}
\end{turn}
\end{textblock}

\end{frame}

\begin{frame}[plain,noframenumbering]

\begin{textblock}{160}(0,0)
\includegraphics[width=1.01\textwidth]{../../aux/static/bali-channel}
\end{textblock}


\begin{textblock}{160}(99,68)
\LARGE \textcolor{black!95}{\rotatebox[origin=tr]{10}{\fontsize{22}{0}\selectfont \textbf{Flujos de}}}
\end{textblock}

\begin{textblock}{160}(103,76)
\LARGE \textcolor{black!95}{\rotatebox[origin=tr]{12}{\fontsize{22}{0}\selectfont \textbf{inferencia}}}
\end{textblock}



\begin{textblock}{55}(48,20)
\begin{turn}{0}
\parbox{15cm}{\textcolor{black!5}{\hspace{0.3cm} Capítulo \unidad} \\
\small \textcolor{black!5}{\hspace{0.7cm} Flujos de \hspace{0.6cm} inferencia} \\
\small \textcolor{black!5}{\hspace{0.3cm} en modelos \hspace{0.7cm} causales.} \\}
\end{turn}
\end{textblock}



\end{frame}


\begin{frame}[plain,noframenumbering]

\begin{textblock}{160}(0,0)
\includegraphics[width=1\textwidth]{../../aux/static/peligro_predador}
\end{textblock}

\begin{textblock}{160}(127,67)
\LARGE \textcolor{black!5}{\fontsize{22}{0}\selectfont \textbf{Inferencia  \\[-0.1cm] \hspace{0.5cm} causal}}
\end{textblock}

\begin{textblock}{55}(2,3)
\begin{turn}{0}
\parbox{15cm}{\small \textcolor{black!95}{Conclusiones causales a partir de datos observacionales. El} \\
\textcolor{black!95}{efecto de las intervenciones sobre los modelos gráficos. Los} \\
\textcolor{black!95}{criterios de puerta trasera y delantera. Contrafácticos.} \\
\normalsize\textcolor{black!95}{Capítulo \unidad} \\
}
\end{turn}
\end{textblock}


\end{frame}


\begin{frame}[plain,noframenumbering]
% \begin{textblock}{160}(0,-80)  \centering
% \includegraphics[width=1\textwidth]{../../aux/static/galton_box}
% \end{textblock}

\begin{textblock}{160}(0,11)  \centering
\includegraphics[width=0.42\textwidth]{../../aux/static/treeOfLife}
\end{textblock}

\begin{textblock}{160}(0,3) \centering
\LARGE \textcolor{black!85}{\rotatebox[origin=tr]{0}{\fontsize{22}{0}\selectfont \textbf{Distribuciones de creencias}}}
\end{textblock}
% % <
% \begin{textblock}{160}(0,3) \centering
% \begin{tikzpicture}
%   \node (Start) at (2.8,0) {};
%   \node (End) at (-2.8,0) {};
%   \draw [decorate,decoration={text along path,text align=center,text={sssssssssssssssssssssssssssssssssss|\bf\fontsize{22}{22}\selectfont|Distribuciones de creencias},text color=black!85 }] (End) to [bend left=45] (Start);
% \end{tikzpicture}
% \end{textblock}

%  \begin{textblock}{160}(0,3) \centering
% \tikz{
% \node[factor, xshift=-3cm, opacity=0] (a) {} ;
% \node[factor, xshift=3cm, opacity=0] (b) {} ;
% \path[draw, -, fill=black!50,sloped,draw opacity=0] (a) edge[bend left=45,draw=black!50] node[color=black!75] {\scriptsize  \texttt{lhood\_lose\_tb}} (b);
% }
% \end{textblock}


\begin{textblock}{55}(75,40)
\textcolor{black!85}{\small El árbol de la vida \\
\fontsize{2}{0}\selectfont Synthesis of phylogeny and taxonomy into a comprehensive tree of life \\}
\end{textblock}


\begin{textblock}{55}(3,82)
\textcolor{black!85}{Capítulo \unidad}
\end{textblock}

\begin{textblock}{55}(23,80)
\begin{turn}{0}
\parbox{15cm}{\small \textcolor{black!85}{Máxima entropía. Gases. Distribución de la riqueza. Procesos irreversibles. Polya Urn.   \\
La familia exponencial: Bernoulli, Binomial, Beta, Multinomial, Dirichlet, Guassiana.} \\
}
\end{turn}
\end{textblock}

\end{frame}

\begin{frame}[plain,noframenumbering]
\begin{textblock}{160}(0,14) \centering
\includegraphics[width=0.8\textwidth]{../../aux/static/biomass.jpg}
\end{textblock}

\begin{textblock}{160}(0,3) \centering
\LARGE \textcolor{black!90}{\fontsize{22}{0}\selectfont \textbf{Evaluación de modelos}}
\end{textblock}

\begin{textblock}{160}(35,23)
\textcolor{black!95}{\small Biomasa de la vida \\
\fontsize{2}{0}\selectfont \hspace{0.05cm} Bar-on et al. The biomass distribution on Earth (2018) \\}
\end{textblock}

\begin{textblock}{160}(16,14)
\LARGE \textcolor{black!0}{\fontsize{1200}{1200}\selectfont $\bm{\bullet}$ }
\end{textblock}
\begin{textblock}{160}(16,16)
\LARGE \textcolor{black!0}{\fontsize{1200}{1200}\selectfont $\bm{\bullet}$ }
\end{textblock}


\begin{textblock}{160}(0,70) \centering
\textcolor{black!95}{Capítulo \unidad \\ \small
La emergencia del sobreajuste (\emph{overfitting}) en los enfoques que seleccionan una única hipótesis. \\
El balance natural de la evaluación de modelo por integración del espacio de hipótesis (evidencia). \\
La forma correcta de evaluar modelo. Ejemplo: regresión líneal bayesiana.\\
}
\end{textblock}


\end{frame}

\begin{frame}[plain,noframenumbering]

\begin{textblock}{160}[0,0](0,-47.5)
\includegraphics[width=1\textwidth]{../../aux/static/raices}
\end{textblock}

% \begin{textblock}{160}[0,1](0,80)
% \includegraphics[width=0.19\textwidth]{../../aux/static/aproximacion1}
% \hspace{-0.1cm}
% \includegraphics[width=0.19\textwidth]{../../aux/static/aproximacion2}
% \hspace{-0.1cm}
% \includegraphics[width=0.19\textwidth]{../../aux/static/aproximacion3}
% \hspace{-0.1cm}
% \includegraphics[width=0.19\textwidth]{../../aux/static/aproximacion4}
% \hspace{-0.1cm}
% \includegraphics[width=0.19\textwidth]{../../aux/static/aproximacion5}
% \end{textblock}

\begin{textblock}{160}(107,3)
\LARGE \textcolor{black!65}{\fontsize{22}{0}\selectfont \textbf{Aproximaciones \\ analíticas}}
\end{textblock}



\begin{textblock}{160}(2,2)
\textcolor{black!80}{Capítulo \unidad \\ \small
Métodos eficientes de aproximación: \\
expectation propagation y variational \\
inference. Ejemplo: estimación de habi-\\ \hspace{0.8cm} lidad en la industria del video juego. \\
}
\end{textblock}

\end{frame}



\begin{frame}[plain,noframenumbering]
\begin{textblock}{160}(0,-4.3) \centering
\includegraphics[width=1\textwidth]{../../aux/static/antartic}
\end{textblock}

\begin{textblock}{160}(0,0) \centering
\tikz{
\node[det, fill=black,draw=black] (k) {\textcolor{black}{--------------------------------------------------------------------------------------------------------------------------------------}} ;
}
\end{textblock}

\begin{textblock}{160}(5,0)
\tikz{
\node[det, fill=black,draw=black,text width=0.01cm] (k) {\textcolor{black}{--------------------------------------------------------------------------------------------------------------------------------------}} ;
}
\end{textblock}


\begin{textblock}{160}(0,4) \centering
\LARGE \hspace{1cm} \textcolor{black!10}{\fontsize{22}{0}\selectfont \textbf{Series de tiempo}}
\end{textblock}


\begin{textblock}{55}[0,1](8,70)
\begin{turn}{90}
\parbox{6cm}{\footnotesize
\textcolor{black!10}{Millones de km$^2$ de hielo Antártico}}
\end{turn}
\end{textblock}


\begin{textblock}{160}(20,63)
\textcolor{black!5}{Capítulo \unidad \\ \small
El problema de usar el posterior como \\
prior del siguiente evento. La mutua dependencia \\
de las hipótesis en modelos de historia completa.  \\
Ejemplo: estimación de habilidad estado-del-arte. \\
}
\end{textblock}


\end{frame}

\begin{frame}[plain,noframenumbering]
\begin{textblock}{160}(-5,0) \centering
\includegraphics[width=1.05\textwidth]{../../aux/static/pajarosTrayectorias}
\end{textblock}
\begin{textblock}{160}(4,20)
\LARGE \textcolor{black!6}{\fontsize{22}{0}\selectfont \textbf{Aproximaciones}}
\end{textblock}
\begin{textblock}{160}(14,27)
\LARGE \textcolor{black!6}{\fontsize{22}{0}\selectfont \textbf{por exploración}}
\end{textblock}


\begin{textblock}{160}(90,44)
\textcolor{black!5}{Capítulo \unidad \\ \small
\hspace{0.4cm} Métodos para modelos causales intratables: \\
\hspace{1.1cm} Markov chain Monte Carlo. Metrópolis \\
\hspace{1.8cm} -Hasting y Hamiltonian MC. \\
}
\end{textblock}


\end{frame}

\begin{frame}[plain,noframenumbering]
\begin{textblock}{160}(0,0) \centering
\includegraphics[width=1.2\textwidth]{../../aux/static/ppls}
\end{textblock}
\begin{textblock}{160}(8,8)
\LARGE \textcolor{black!15}{\fontsize{22}{0}\selectfont \textbf{Programación \\ probabilistica \\}}
\end{textblock}
% \begin{textblock}{160}(94,23)
% \LARGE \textcolor{black!16}{\fontsize{22}{0}\selectfont \textbf{probabilistica}}
% \end{textblock}

\begin{textblock}{160}(70,30)
\normalsize
\textcolor{black!25}{
\tikz{
\node[det, fill=black!60,draw=black!25] (k) {\textcolor{black!5}{$k_i$}} ;
\node[latent, fill opacity=0, draw=black!25, text opacity=1, above=of k, xshift=-1cm] (p) {\textcolor{black!5}{$p$}};
\node[det, fill opacity=0, draw=black!25, text opacity=1, above=of k, xshift=1cm] (n) {\textcolor{black!5}{$n$}};
\edge {p,n} {k};
\plate[inner sep=0.3cm, xshift=0cm, yshift=0.12cm] {intentos} {(k)} {$i$}
\node[const, right=of n, xshift=0.3cm] (np) {$p \sim \text{Beta}(1,1)$};
\node[const, right=of n, xshift=0.3cm, yshift=-1cm] (np) {$n \sim \text{Categorical}(N_\text{max})$};
\node[const, right=of n, xshift=0.3cm, yshift=-2cm] (np) {$k_i \sim \text{Binomial}(p,n)$};
}
}
\end{textblock}

\begin{textblock}{160}(20,74)
\textcolor{black!15}{Capítulo \unidad \\ \small
Implementación de modelos usando lenguajes de programación probabilística. \\
Verificación visual de buen funcionamiento de las aproximaciones.  \\
}
\end{textblock}


\end{frame}

\begin{frame}[plain,noframenumbering]

\begin{textblock}{96}(0,7)\centering
{\transparent{0.1}\includegraphics[width=0.8\textwidth]{../../aux/static/inti.png}}
\end{textblock}

\begin{textblock}{160}(96,00)
\includegraphics[width=0.4\textwidth]{../../aux/static/pachacuteckoricancha}
\end{textblock}

\begin{textblock}{90}(03,4) \scriptsize
\parbox{9cm}{A diferencia de las ciencias formales, que validan sus proposiciones dentro de sistemas axiomáticos cerrados, las ciencias empíricas (desde la física hasta las ciencias sociales) deben validar sus proposiciones en sistemas abiertos que por definición contienen siempre algún grado de incertidumbre. ¿Es posible alcanzar ``verdades'' si es inevitable decir ``no sé''? Sí. La aplicación estricta de las reglas de la probabilidad (enfoque Bayesiano) garantiza los acuerdos intersubjetivos en contextos de incertidumbre, fundamento de las verdades empíricas. Bajo incertidumbre, la lógica es paraconsistente en tanto se hace necesario creer al mismo tiempo en A y no A hasta que la sorpresa, única fuente de información, decida. Debido a que este proceso de selección es, como el evolutivo, de naturaleza multiplicativa (un solo cero en la secuencia de reproducción y supervivencia genera una extinción), existe una ventaja a favor de las variantes que reducen las fluctuaciones. Si bien la aplicación estricta de la teoría de la probabilidad ha mostrado ser la lógica ideal en contextos de incertidumbre, su adopción se vio históricamente limitada debido al alto costo computacional asociado. A diferencia del enfoque frecuentista de la probabilidad que selecciona una única hipótesis, el enfoque Bayesiano actualiza las creencias de todas y cada una de las hipótesis de acuerdo a la evidencia empírica y formal (datos y modelos). Si bien en las últimas décadas las limitaciones computacionales han sido superadas en gran medida gracias al desarrollo de métodos eficientes de aproximación, la inercia histórica es ahora su limitación principal. Este curso tiene por objetivo promover la adopción del enfoque Bayesiano como método general para la resolución de cualquier problema empírico: en la ciencia, la política y la ecología.}
\end{textblock}

\end{frame}

\end{document}
