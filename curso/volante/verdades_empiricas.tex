\documentclass[shownotes,aspectratio=169]{beamer}

\input{../../aux/tex/diapo_encabezado.tex}
\input{../../aux/tex/tikzlibrarybayesnet.code.tex}
 \mode<presentation>
 {
 %   \usetheme{Madrid}      % or try Darmstadt, Madrid, Warsaw, ...
 %   \usecolortheme{default} % or try albatross, beaver, crane, ...
 %   \usefonttheme{serif}  % or try serif, structurebold, ...
  \usetheme{Antibes}
  \setbeamertemplate{navigation symbols}{}
 }

\usepackage{rotating}

\usepackage{todonotes}
\setbeameroption{show notes}


\estrue

%\title[Bayes del Sur]{}

\begin{document}

\color{black!85}
\large

\begin{frame}[plain,noframenumbering]


\begin{textblock}{160}(0,0)
\includegraphics[width=1\textwidth]{../../aux/static/deforestacion}
\end{textblock}

\begin{textblock}{80}(18,9)
\textcolor{black!15}{\fontsize{44}{55}\selectfont Verdades}
\end{textblock}

\begin{textblock}{47}(85,70)
\centering \textcolor{black!15}{{\fontsize{52}{65}\selectfont Empíricas}}
\end{textblock}

\begin{textblock}{80}(100,28)
\LARGE  \textcolor{black!15}{\rotatebox[origin=tr]{-3}{\scalebox{9}{\scalebox{1}[-1]{$p$}}}}
\end{textblock}


\begin{textblock}{80}(66,43)
\LARGE  \textcolor{black!15}{\scalebox{6}{$=$}}
\end{textblock}

\begin{textblock}{80}(36,29)
\LARGE  \textcolor{black!15}{\scalebox{9}{$p$}}
\end{textblock}

\vspace{2cm}
\maketitle



\begin{textblock}{160}(01,67)
\normalsize \textcolor{black!5}{\textbf{Acuerdos inte$\dots \ \ \ \ \ \ \ \ \ \ \ \   \dots$umbre}}
\end{textblock}

% Lugar
\begin{textblock}{160}(01,73)
\scriptsize \textcolor{black!5}{
Curso ``Metodologías del conocimiento empírico''\\
Congreso Bayesiano de las Provincias Unidas\\
Taller Argentino de Computación Científica \\
Encuentro anual del 4 al 7 de agosto 2023 \\
La Banda, Santiago del Estero, Argentina
}
\end{textblock}

\end{frame}


\begin{frame}[plain,noframenumbering]
\begin{textblock}{170}(-9,0)
\rotatebox[origin=tr]{90}{\includegraphics[width=0.53\textwidth]{../../aux/static/egipto3.jpeg}}
\end{textblock}

\begin{textblock}{160}(16,9)
\LARGE \textcolor{black!5}{\fontsize{22}{0}\selectfont \textbf{Principios interculturales}}
\end{textblock}
\begin{textblock}{160}(22,18)
\LARGE \textcolor{black!5}{\fontsize{22}{0}\selectfont \textbf{de acuerdos intersubjetivos}}
\end{textblock}


\begin{textblock}{55}(71,38)
\begin{turn}{30}
\parbox{6cm}{
\textcolor{black!5}{\hspace{-0.3cm}Capítulo 1} \\
\small\textcolor{black!5}{\hspace{-0.1cm}Principio de razón suficiente, de}\\
\small\textcolor{black!5}{integridad, de indiferencia y de} \\
\small\textcolor{black!5}{\hspace{0.1cm}coherencia. Las reglas de razo-} \\ \small\textcolor{black!5}{\hspace{0.2cm}namiento bajo incertidumbre.} \\
\small\textcolor{black!5}{\hspace{0.3cm}Evaluación de creencias.} \\
}
\end{turn}
\end{textblock}

\end{frame}

\begin{frame}[plain,noframenumbering]

% \begin{textblock}{160}(0,0)
% \includegraphics[width=1.18\textwidth]{../../aux/static/fotosintesis}
% \end{textblock}
\begin{textblock}{160}(0,-15)
\includegraphics[width=1\textwidth]{../../aux/static/tsimane}
\end{textblock}


% VERSION 2
\begin{textblock}{160}(6,36)
\LARGE \rotatebox[origin=tr]{18}{\textcolor{black!95}{\fontsize{22}{0}\selectfont \textbf{La función}}}
\end{textblock}
\begin{textblock}{160}(41,32)
\LARGE \rotatebox[origin=tr]{23}{\textcolor{black!95}{\fontsize{22}{0}\selectfont \textbf{de}}}
\end{textblock}
\begin{textblock}{160}(50.5,23)
\LARGE \rotatebox[origin=tr]{28}{\textcolor{black!95}{\fontsize{22}{0}\selectfont \textbf{costo}}}
\end{textblock}
\begin{textblock}{160}(68,5.3)
\LARGE \rotatebox[origin=tr]{26}{\textcolor{black!95}{\fontsize{22}{0}\selectfont \textbf{epistémico}}}
\end{textblock}
\begin{textblock}{160}(104,5.5)
\LARGE \rotatebox[origin=tr]{8}{\textcolor{black!95}{\fontsize{22}{0}\selectfont \textbf{-}}}
\end{textblock}
\begin{textblock}{160}(110,3)
\LARGE \rotatebox[origin=tr]{-14}{\textcolor{black!95}{\fontsize{22}{0}\selectfont \textbf{evolutiva}}}
\end{textblock}


\begin{textblock}{55}[0,0](130,25)
\begin{turn}{-90}
\parbox{7cm}{\sloppy\setlength\parfillskip{0pt}
\textcolor{black!0}{\ \ \ \ \ Capítulo 2} \\
\small\textcolor{black!5}{\ \ \ \ Ventajas a favor de la:} \\
\small\textcolor{black!5}{ \ \ Diversificación (propiedad epistémica)\\ Cooperación (propiedad evolutiva mayor)}\\
\small\textcolor{black!5}{ \hspace{-0.15cm}Especialización (propiedad meta-epistémica)} \\
\small\textcolor{black!5}{\hspace{-0.45cm} Coexistencia (propiedad ecológica).\\ }}
\end{turn}
\end{textblock}


\end{frame}

\begin{frame}[plain,noframenumbering]

\begin{textblock}{160}(0,0)
\includegraphics[width=1\textwidth]{../../aux/static/fuego}
\end{textblock}

\begin{textblock}{160}(4,26)
\LARGE \textcolor{black!5}{\fontsize{22}{0}\selectfont \textbf{Sorpresa: el problema}}
\end{textblock}
\begin{textblock}{160}(4,34)
\LARGE \textcolor{black!5}{\fontsize{22}{0}\selectfont \textbf{de la comunicación}}
\end{textblock}
\begin{textblock}{160}(4,42)
\LARGE \textcolor{black!5}{\fontsize{22}{0}\selectfont \textbf{con la realidad}}
\end{textblock}
% \begin{textblock}{160}(3,82)
% \LARGE \textcolor{black!15}{\fontsize{22}{0}\selectfont \textbf{3}}
% \end{textblock}



\begin{textblock}{55}[0,0](88,25)
\begin{turn}{0}
\parbox{7cm}{\sloppy\setlength\parfillskip{0pt}
\textcolor{black!0}{Capítulo 3} \\
\small\textcolor{black!5}{\hspace{0.05cm}La estructura invariante del dato empírico:} \\
\small\textcolor{black!5}{\hspace{0.1cm}fuente, realidad causal, señal, canal,} \\ \small\textcolor{black!5}{\hspace{0.05cm}percepción, modelo causal, estimación.} \\
\small\textcolor{black!5}{\hspace{-0.15cm}Base empírica y datos teóricos. Máxima} \\
\small\textcolor{black!5}{\hspace{-0.35cm}incertidumbre y mínima sorpresa. Información.} \\
}
\end{turn}
\end{textblock}


\end{frame}

\begin{frame}[plain,noframenumbering]
\begin{textblock}{160}(0,43)
\includegraphics[width=1\textwidth]{../../aux/static/modelosGraficos}
\end{textblock}


\begin{textblock}{160}(4,8)
\LARGE \textcolor{black!85}{\fontsize{22}{0}\selectfont \textbf{Modelos gráficos y}}
\end{textblock}
\begin{textblock}{160}(4,16)
\LARGE \textcolor{black!85}{\fontsize{22}{0}\selectfont \textbf{algoritmo suma-producto}}
\end{textblock}


\begin{textblock}{55}[0,0](70,27)
\begin{turn}{0}
\parbox{10cm}{\sloppy\setlength\parfillskip{0pt}
\textcolor{black!85}{Capítulo 4} \\
\small\textcolor{black!85}{Métodos gráficos de especificación de modelos causales.} \\
\small\textcolor{black!85}{Cómputo descentralizado de la inferencia y la predicción:} \\
\small\textcolor{black!85}{pasaje de mensajes entre los nodos de las redes causales}\\
}
\end{turn}
\end{textblock}

\end{frame}

\begin{frame}[plain,noframenumbering]

\begin{textblock}{160}(0,0)
\includegraphics[width=1.01\textwidth]{../../aux/static/bali-channel}
\end{textblock}


\begin{textblock}{160}(99,68)
\LARGE \textcolor{black!95}{\rotatebox[origin=tr]{10}{\fontsize{22}{0}\selectfont \textbf{Flujos de}}}
\end{textblock}

\begin{textblock}{160}(103,76)
\LARGE \textcolor{black!95}{\rotatebox[origin=tr]{12}{\fontsize{22}{0}\selectfont \textbf{inferencia}}}
\end{textblock}



\begin{textblock}{55}(48,20)
\begin{turn}{0}
\parbox{15cm}{\textcolor{black!5}{\hspace{0.3cm} Capítulo 5} \\
\small \textcolor{black!5}{\hspace{0.7cm} Flujos de \hspace{0.6cm} inferencia} \\
\small \textcolor{black!5}{\hspace{0.3cm} en modelos \hspace{0.7cm} causales.} \\}
\end{turn}
\end{textblock}



\end{frame}


\begin{frame}[plain,noframenumbering]

\begin{textblock}{160}(0,0)
\includegraphics[width=1\textwidth]{../../aux/static/peligro_predador}
\end{textblock}

\begin{textblock}{160}(127,67)
\LARGE \textcolor{black!5}{\fontsize{22}{0}\selectfont \textbf{Inferencia  \\[-0.1cm] \hspace{0.5cm} causal}}
\end{textblock}

\begin{textblock}{55}(2,3)
\begin{turn}{0}
\parbox{15cm}{\small \textcolor{black!95}{Conclusiones causales a partir de datos observacionales. El} \\
\textcolor{black!95}{efecto de las intervenciones sobre los modelos gráficos. Los} \\
\textcolor{black!95}{criterios de puerta trasera y delantera. Contrafácticos.} \\
\normalsize\textcolor{black!95}{Capítulo 6} \\
}
\end{turn}
\end{textblock}


\end{frame}


\begin{frame}[plain,noframenumbering]
% \begin{textblock}{160}(0,-80)  \centering
% \includegraphics[width=1\textwidth]{../../aux/static/galton_box}
% \end{textblock}

\begin{textblock}{160}(0,11)  \centering
\includegraphics[width=0.42\textwidth]{../../aux/static/treeOfLife}
\end{textblock}

\begin{textblock}{160}(0,3) \centering
\LARGE \textcolor{black!85}{\rotatebox[origin=tr]{0}{\fontsize{22}{0}\selectfont \textbf{Distribuciones de creencias}}}
\end{textblock}

\begin{textblock}{55}(3,82)
\textcolor{black!85}{Capítulo 7}
\end{textblock}

\begin{textblock}{55}(23,80)
\begin{turn}{0}
\parbox{15cm}{\small \textcolor{black!85}{Máxima entropía. Gases. Distribución de la riqueza. Procesos irreversibles. Polya Urn.   \\
La familia exponencial: Bernoulli, Binomial, Beta, Multinomial, Dirichlet, Guasiana.} \\
}
\end{turn}
\end{textblock}

\end{frame}

\begin{frame}[plain,noframenumbering]
\begin{textblock}{160}(0,14) \centering
\includegraphics[width=0.8\textwidth]{../../aux/static/biomass.jpg}
\end{textblock}

\begin{textblock}{160}(0,3) \centering
\LARGE \textcolor{black!90}{\fontsize{22}{0}\selectfont \textbf{Evaluación de modelos}}
\end{textblock}

\begin{textblock}{160}(33,23)
\large \textcolor{black!95}{Biomasa de la vida}
\end{textblock}

\begin{textblock}{160}(16,14)
\LARGE \textcolor{black!0}{\fontsize{1200}{1200}\selectfont $\bm{\bullet}$ }
\end{textblock}
\begin{textblock}{160}(16,16)
\LARGE \textcolor{black!0}{\fontsize{1200}{1200}\selectfont $\bm{\bullet}$ }
\end{textblock}


\begin{textblock}{160}(0,70) \centering
\textcolor{black!95}{Capítulo 8 \\ \small
No todo da lo mismo: la forma correcta de evaluar modelo. \\[0.05cm]
La emergencia del sobreajuste (\emph{overfitting}) en los enfoques seleccionan hipótesis optimizan . \\
El balance natural de la evaluación de modelo por integración del espacio de hipótesis (evidencia).
}
\end{textblock}


\end{frame}

\begin{frame}[plain,noframenumbering]

\begin{textblock}{160}[0,0](0,-47.5)
\includegraphics[width=1\textwidth]{../../aux/static/raices}
\end{textblock}

% \begin{textblock}{160}[0,1](0,80)
% \includegraphics[width=0.19\textwidth]{../../aux/static/aproximacion1}
% \hspace{-0.1cm}
% \includegraphics[width=0.19\textwidth]{../../aux/static/aproximacion2}
% \hspace{-0.1cm}
% \includegraphics[width=0.19\textwidth]{../../aux/static/aproximacion3}
% \hspace{-0.1cm}
% \includegraphics[width=0.19\textwidth]{../../aux/static/aproximacion4}
% \hspace{-0.1cm}
% \includegraphics[width=0.19\textwidth]{../../aux/static/aproximacion5}
% \end{textblock}

\begin{textblock}{160}(107,3)
\LARGE \textcolor{black!65}{\fontsize{22}{0}\selectfont \textbf{Aproximaciones \\ analíticas}}
\end{textblock}



\begin{textblock}{160}(2,2)
\textcolor{black!80}{Capítulo 9 \\ \small
Métodos eficientes de aproximación: \\
expectation propagation y variational \\
inference. Ejemplo: estimación de habi-\\ \hspace{0.8cm} lidad en la industria del video juego. \\
}
\end{textblock}

\end{frame}



\begin{frame}[plain,noframenumbering]
\begin{textblock}{160}(0,-4.3) \centering
\includegraphics[width=1\textwidth]{../../aux/static/antartic}
\end{textblock}
\begin{textblock}{160}(45,70) \centering
\LARGE \textcolor{black!16}{\fontsize{22}{0}\selectfont \textbf{Series de tiempo}}
\end{textblock}


\begin{textblock}{160}(20,58)
\textcolor{black!5}{Capítulo 10 \\[0.1cm] \small
El problema de usar el posterior como \\
prior del siguiente evento. La mutua dependencia \\
de las hipótesis en modelos de historia completa.  \\
Ejemplo: estimación de habilidad estado-del-arte. \\
}
\end{textblock}


\end{frame}

\begin{frame}[plain,noframenumbering]
\begin{textblock}{160}(-5,0) \centering
\includegraphics[width=1.05\textwidth]{../../aux/static/pajarosTrayectorias}
\end{textblock}
\begin{textblock}{160}(4,20)
\LARGE \textcolor{black!6}{\fontsize{22}{0}\selectfont \textbf{Aproximaciones}}
\end{textblock}
\begin{textblock}{160}(14,27)
\LARGE \textcolor{black!6}{\fontsize{22}{0}\selectfont \textbf{por exploración}}
\end{textblock}


\begin{textblock}{160}(100,38)
\textcolor{black!5}{Capítulo 11 \\ \small
El problema de usar el posterior como \\
prior del siguiente evento. Modelos de \\
historia completa. Mutua dependencia de las hipótesis \\
y algoritmo de convergencia. Modelo estado-del-arte de estimación de habilidad. \\
}
\end{textblock}


\end{frame}

\begin{frame}[plain,noframenumbering]
\begin{textblock}{160}(0,0) \centering
\includegraphics[width=1.02\textwidth]{../../aux/static/ppls_stan}
\end{textblock}
\begin{textblock}{160}(62,8)
\LARGE \textcolor{black!25}{\fontsize{22}{0}\selectfont \textbf{Programación probabilistica}}
\end{textblock}
% \begin{textblock}{160}(94,23)
% \LARGE \textcolor{black!16}{\fontsize{22}{0}\selectfont \textbf{probabilistica}}
% \end{textblock}


\end{frame}

\begin{frame}[plain,noframenumbering]

\begin{textblock}{160}(96,0)
\includegraphics[width=0.4\textwidth]{../../aux/static/pachacuteckoricancha}
\end{textblock}


\begin{textblock}{80}(151,3)
\LARGE  \textcolor{black!85}{\rotatebox[origin=tr]{-3}{\scalebox{1.6}{\scalebox{1}[-1]{$p$}}}}
\end{textblock}
\begin{textblock}{80}(151,3)
\LARGE  \textcolor{black!85}{\scalebox{1.6}{$p$}}
\end{textblock}


\end{frame}

\end{document}
